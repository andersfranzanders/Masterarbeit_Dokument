\chapter{Grundlagen der Signalverarbeitung}

Ein \emph{Signal} ist eine Funktion eines Parameters mit numerischen Wertebereich. Die Abbildung zwischen Defintions- und Wertebereich kann, aber muss nicht durch eine Formel definiert sein. So fällt $f(x) = \sin( x )$ genauso unter die Definition eines Signals wie eine Folge numerischer Werte, die durch die Aufnahme eines Messgerätes enstanden sind. Weiterhin kommt dem Wertebereich eine gewissen Bedeutung zu, wie \emph{Zeit} oder \emph{Ort}. Ein typisches Beispiel für ein Signal ist die Spannung, die abhängig von der Zeit von einem Mikrofon erzeugt wird.  Da in dieser Arbeit nur Signale von Bedeutung sind, deren Wertebereich sich auf die Zeit bezieht, konzetrieren sich alle folgenden Bereich auf diesen Bereich- Im Zusammenhang mit Signalen wird der Definitionsbereich auch als \emph{unabhängiger Parameter} und der Wertebereich auch als \emph{abhängiger Parameter} bezeichnet. \cite[S. 11-12]{dspGuide} \cite[S. 22-23]{dspMichigan}

%\emph{ } 
\medskip

 Bei einem zeit-kontinuierlichen Signal $x( t )$ ist der Wertebereich kontinuierlich, wie in Formel  \ref{eq:time-cont-signal} definiert. Bei einem zeit-diskreten Signal $x[n]$ ist der Wertebreich diskret, wie in Formel \ref{eq:time-disc-signal} definiert. So beschreibt beispielsweise $x[17] = s$ den Wert zur Zeit $n = 17$. \glqq Zeit\grqq{} hat in diesem Kontext keine Einheit. Ein Wert wird auch als \emph{Sample} oder \emph{Amplitude} bezeichnet. $x[17] $ meint somit das 17. Sample des Signals. Abbildung \ref{img:aSignal} zeigt Beispiele für ein zeit-kontinuierliches und ein zeit-diskretes Signal. \cite[S. 22 - 23]{dspMichigan}

 \begin{equation}
x(t) = s \; , t \in \mathbb{R}
\label{eq:time-cont-signal}
\end{equation}


\begin{equation}
x[n] = s \; , n \in \mathbb{Z} 
\label{eq:time-disc-signal}
\end{equation}

\begin{figure}
	\centering
	\includegraphics[width=0.7\textwidth]{bilder/aSignal02.png}
	\caption{Ein zeit-kontinuierliches Signal (A) und ein zeit-diskretes Signal (B)}
	\label{img:aSignal}
\end{figure}

Zeit-diskrete Signale werden häufig dadurch gewonnen, dass ein zeit-kontinuierliches Signal in regelmäßigen Intervallen abgetastet wird. Dieser Prozess wird als \emph{Sampling} bezeichnet und durch Formel \ref{eq:sampling} definiert. Der Parameter $T_s$ wird als \emph{Sampling-Interval} bezeichnet. Das Reziproke $\frac{1}{T_s} = f_s$ heißt \emph{Sampling-Rate} und wird in der Einheit $\frac{1}{\text{s}} = \text{Hz}$. Eine Sampling-Rate von $f_s = \SI{44100}{\hertz}$ bedeutete beispielsweise, dass ein Signal 44100 mal pro Sekunde abgetastet wurde.\cite[S. 24]{dspMichigan}

\begin{equation}
x[n] = s(n \cdot T_s) \; , -\infty < n < \infty
\label{eq:sampling}
\end{equation}
	
Da in dieser Arbeit nur zeit-diskrete Signale von Interesse sind, werden ab diesem Punkt die Definitonen für zeit-kontinuierliche Signale ausgelassen. Der \emph{Support} ist das kleinst mögliche Zeitintervall, der alle Samples enthält, die nicht den Wert 0 haben, wie Formel \ref{eq:support} definiert. Die \emph{Dauer} eines Signales ist die Länge des Supportes nach Formel \ref{eq:duration}. Das Signal $x[n] = \cos(n) \: ,0\leq n \leq 3$ hat beispielsweise den Support $[0,3] = \{0,1,2,3\} $ und die Dauer $4$. Ein \emph{unendliches Signal} hat einen unendlichen langen Support, das heißt es gilt Duration$(x) = \infty$. Ein \emph{endliches Signal} hat einen endlichen Support, das heißt Duration$(x) \neq\infty$. Unabhängig von der Endlichkeit oder Unendlichkeit des Supportes wird davon ausgegangen, dass sich alle Signale von negativer bis positiver Unendlichkeit erstrecken. Werden also berechnungen auf Samples eines Signales durchgeführt, die außerhalb seines Supportes liegen, werden diese Samples mit dem Wert 0 angenommen. \cite[S. 24]{dspMichigan}

\begin{equation}
\label{eq:support}
\begin{split}
\text{Sup}(x) = [sup_s, sup_e] \quad , sup_s, sup_e \in \mathbb{Z} \\,  x[sup_s] \neq 0 \:  \wedge \:  x[sup_e] \neq 0 \: \wedge \: \forall n \
\not\in [sup_s, sup_e] : x[n] = 0
\end{split}
\end{equation}

\begin{equation}
\text{Duration}(x) = sup_e - sup_s + 1
\label{eq:duration}
\end{equation}

Ein Signal gilt als \emph{periodisch}, wenn Formel \ref{eq:periodicity} erfüllt ist. Der Parameter $N$ wird als \text{Periode} von $x$ bezeichnet. Wenn ein Signal mit $N$ periodisch ist, dann ist es auch mit $2N, 3N, \ldots $ periodisch. Die Grundfrequenz $N_0$ ist das kleinste N, für das Formel \ref{eq:periodicity} erfüllt ist. Abbildung \ref{img:periodicSic} zeigt ein Beispiel für ein nicht-periodisches und ein periodisches Signal. \cite[S. 24]{dspMichigan}

\begin{equation}
\exists N : \forall n \in Sup : x[n+N] = x[n] \rightarrow \text{Periodisch}(x,N) = true
\label{eq:periodicity}
\end{equation}

\begin{figure}[h]
	\centering
	\includegraphics[width=0.7\textwidth]{bilder/periodicSig.png}
	\caption{Ein nicht-periodisches Signal (A) und ein periodisches Signal (B)}
	\label{img:periodicSic}
\end{figure}

\section{Statistische Merkmale}

Im folgenden wird ein überblick über die häufig verwendete Signaleigenschaften gegeben. Abbildung \ref{img:sigStats} visualisiert die Erläuterungen.

\begin{enumerate}[leftmargin=*]
	
\item Der \textbf{Maximalwert / Minimalwert} beschreibt den höchsten / niedrigsten in  $x$ enthaltenen Wert nach den Formel $\max(x)$ und $\min(x)$.
	
\item Der \textbf{Durchschnittswert / Average Value} beschreibt den durchschnittlichen Wert aller Samples von $x$ nach Formel \ref{eq:avg}. Dieser Durchschnittswert wird über dem Intervall $[n_1, n_2]$  berechnet.

\begin{equation}
\text{AVG}(x) = \frac{1}{n_2 - n_1 + 1} \sum_{n = n_1}^{n_2} x[n]
\label{eq:avg}
\end{equation}

\item Der \textbf{Mean Squared Value} (\emph{MSV}) beschreibt den quadrierten Durchschnittswert über eine bestimmtes Interval nach Formel \ref{eq:msv}. Er wird auch als \emph{durchschnittliche Energie} oder \emph{average Power} bezeichnet.

\begin{equation}
\text{MSV}(x) = \frac{1}{n_2 - n_1 + 1} \sum_{n = n_1}^{n_2} x[n]^2
\label{eq:msv}
\end{equation}

\item Das \textbf{Root Mean Square} (\emph{RMS}) ist die Wurzel des Mean Squared Value nach Formel\ref{eq:rms}. Der RMS findet häufiger Anwendung als der MSV, da er besser ins Verhältnis zu den Werten des Signals gesetzt werden kann. Er wird im Deutschen auch als \textbf{Effektivwert} oder \textbf{Durchschnittsleistung} bezeichnet. Da die deutschen Begriffe in einigen Quellen jedoch auch für den MSV verwendet werden, wird an dieser Stelle nur mit den englischen Begriffen gearbeitet.

\begin{equation}
\text{RMS}(x) = \sqrt{\frac{1}{n_2 - n_1 + 1} \sum_{n = n_1}^{n_2} x[n]^2}
\label{eq:rms}
\end{equation}

\item Die \textbf{Energie / Energy} bezeichnet die \glqq Stärke \grqq{} eines Signals über einen bestimmten Intervall nach Formel \ref{eq:energy}. Sie entspricht dem MSV-Wert multipliziert der Länge des Intervalls. \cite[S. 27-28]{dspMichigan}

\begin{equation}
\text{E}(x) = \sum_{n = n_1}^{n_2} x[n]^2
\label{eq:energy}
\end{equation}
	
\end{enumerate}	

\begin{figure}[h]
	\centering
	\includegraphics[width=0.7\textwidth]{bilder/sigStats.png}
	\caption{Statistische Werte eines Signals über das Intervall [50,200]}
	\label{img:sigStats}
\end{figure}

Die Addition und Multiplikation wird bei Signalen komponentenweise durchgeführt, das heißt $x_1[n] + x_2[n] = y[n] $ und $x_1[n] \cdot x_2[n] = y[n] $. Abbildung \ref{img:addAndMultSig} visualisiert diese Operationen. 

\begin{figure}[h]
	\centering
	\includegraphics[width=0.8\textwidth]{bilder/addAndMultSig.png}
	\caption{Komponentenweise Addition und Mulitplikation zweier Signale}
	\label{img:addAndMultSig}
\end{figure}

\section{Fehlersignale}

Die Addition wird unter anderem für die Modellierung des Einflusses von Störungen benötigt. Angenommen, ein Signal $x$ wird übertragen, auf dem Übertragungsweg jedoch durch ein anderes Störsignal wie z.B. Rauschen $e$ überlagert. Dieses Störsignal wird in diesem Zusammenhang auch als \glqq{Fehler-Signal} bezeichnet. Das resultierende Signal $x'$ wird nach Formel \ref{eq:sigErrorAddition} berechnet. Kennt man sowohl das Eingangssignal $x$ als auch das Ausgangssignal $x'$, kann das Störsignal $e$ nach Formel \ref{eq:calErrorSig} berechnet werden.

\begin{equation}
x'[n] = x[n] + e[n]
\label{eq:sigErrorAddition}
\end{equation}

\begin{equation}
e[n] = x'[n] -x[n]
\label{eq:calErrorSig}
\end{equation}

 Errechnet man nun den den MSV- oder RMS-Wert des Störsignales $e$, gibt das Ergebnis einen Eindrück über die \glqq Stärke \grqq{} des Fehler-Signals. Der MSE-Wert des Fehlers wird in diesem Zusammenhang auch als \emph{Mean Squared Error} (\emph{MSE}) und der RMS-Wert als \emph{Root Mean Squared Error} (\emph{RMSE}) oder einfach als \emph{Fehler} oder \emph{Error} bezeichnet. Formel\ref{eq:mse} und \ref{eq:error} definierten die Berechnungen des MSE und RMSE. Der RMSE hat im Gegensatz zum MSE den Vorteil, dass er besser ins Verhältnis zu den Werten des Fehlersignals gestetzt werden kann. Ein RMSE $= 0$ heisst, dass $x = x'$ und somit kein Störsignal vorliegt. Ein RMSE = RMS$(x)$ heisst, dass Eingangs- und Störsignal den selben Effektivwert und somit die selbe \glqq stärke\grqq{} besitzen. Abbildung \ref{img:snrStuff} visualisiert die Berechnung des MSE und RMSE. \cite[S: 28 - 29]{dspMichigan}

\begin{equation}
\text{MSE}(x,x') = \frac{1}{n_2 - n_1 + 1} \sum_{n = n_1}^{n_2} (x[n]-x'[n])^2
\label{eq:mse}
\end{equation}

\begin{equation}
\text{RMSE}(x,x') = \sqrt{\frac{1}{n_2 - n_1 + 1} \sum_{n = n_1}^{n_2} (x[n]-x'[n])^2}
\label{eq:error}
\end{equation}

Eine weitere Betrachtungsweise bezüglich der Stärke des Rauschens auf das Signal ist, das Eingangssignal ins Verhältnis zum Rauschsignal zu setzen. Formel \ref{eq:snrPre} gibt die Definition. Ein SNR\textsubscript{rel}$(x,e) = 1$ heißt, dass das Eingangssignal den selben MSV wie das Fehlersignal hat. Meistens ist der MSV des Eingangssignals in der Praxis sehr viel höher als der des Fehler-Signals. Um den Zahlenraum zu begrenzen, wird die Pseudo-Einheit dB verwendet. Formel \ref{eq:snrDb} den so berechneten \emph{Signal-Rausch-Abstand} (\emph{SNR}, englisch Signal-to-Noise-Ratio). Entgegen des MSE weisst ein \emph{niedriger} SNR-Wert auf ein \emph{starkes} Rauschen hin, und ein \emph{hoher} SNR auf ein \emph{schwaches} Rauschen! Abbildung \ref{img:snrStuff} visualisiert die Berechnung des SNR.

%% To do: Gute Quelle suchen!!

\begin{equation}
\text{SNR}_{rel}(x,e) = \frac{MSV(x)}{MSV(e)}
\label{eq:snrPre}
\end{equation}

\begin{equation}
\text{SNR}(x,e) = 10 \cdot  \lg \Big(\frac{MSV(x)}{MSV(e)} \Big) \text{ dB}
\label{eq:snrDb}
\end{equation}

\begin{figure}[h]
	\centering
	\includegraphics[width=1\textwidth]{bilder/snrStuff02.png}
	\caption{Berechnung des MSE, RMSE und SNR eines von Rauschen gestörten Signals}
	\label{img:snrStuff}
\end{figure}

\section{Korrelation}

Die \emph{Korrelation} (engl \emph{Correlation}) zweier Signale $x_1$ und $x_2$ wird nach Formel \ref{eq:correlation} als die Summe aller Samples des Produktes der beiden Signale über einen bestimmtes Intervall $[n_1, n_2]$ definiert. Das Ergebnis ist eine Wert $\in \mathbb{R}$ welches die \glqq Ähnlichkeit der beiden Signale\grqq{} kennzeichnet. Ein Positiver Wert weisst auf eine \emph{positive Korrelation} hin, ein negativer Wert auf eine \emph{negative Korrelation}, und ein Wert von $\text{Corr}(x_1,x_2) = 0$ auf \emph{keine Korrelatoin}. Aus der größe des Wertes kann die Stärke der Korrelation jedoch nicht direkt interpretiert werden. Bei der \emph{normalisierten Korrelation} Corr$_N(x,y)$ wird daher die Korrelationswert ins Verhältnis zu den Energien der beiden Signale gesetzt, wie in Formel \ref{eq:normCorrelation} definiert. Der Wertebereich der normalisierten Autokorrelation  ist $-1 \leq \text{Corr}_N(x,y) \leq +1$. Daraus ergeben sich die in Formel \ref{eq:correlationProps} definierten Zusammenhänge. Ein Wert von $ \text{Corr}_N(x,y) = 1$ wird auch als \emph{perfekte Korrelation} bezeichnet, ein Wert von  $ \text{Corr}_N(x,y) = -1$ als \emph{anti-perfekte Korrelation} \cite[S. 46 - 47]{dspMichigan} Abbildung \ref{img:corrSigsComp} visualisiert die normalisierte Korrelation eines Signales $x$ mit den Signalen $y_n$.

\begin{equation}
\text{Corr}(x,y) = \sum_{n=n_1}^{n_2} x[n] \cdot y[n]
\label{eq:correlation}
\end{equation}

\begin{equation}
\text{Corr}_N(x,y) = \frac{\text{Corr}(x,y)}{\sqrt{\text{E}(x) \cdot \text{E}(y)}}
\label{eq:normCorrelation}
\end{equation}

\begin{equation}
\text{Corr}_N(x,y) = 
\begin{cases}
1  \quad \rightarrow  x = y \\
-1 \; \rightarrow x = -y
\end{cases}
\label{eq:correlationProps}
\end{equation}

\begin{figure}[h]
	\centering
	\includegraphics[width=1\textwidth]{bilder/corrSigsComp.png}
	\caption{Correlation der Signale $x$ und $y$}
	\label{img:corrSigsComp}
\end{figure}

Die Korrelation und die normalisierte Korrelation werden aufgrund ihrer Eigenschaften verwendet, um ein Signal $x$ in einem Signal $y$ zu detektieren. Häufig ist das Ziel, ein von einem Rauschen $e$ überlagerten Signal $x+e = y$ auf das Vorhandensein des erwarteten Signales $x$ hin zu überprüfen. Wie in Abbildung \ref{img:corrSigsComp} zu sehen ist, ist der Korrelationswert jedoch von der Verzögerung des Signals abhängig. Daher wird in der $Cross-Correlation$ das Signal $y$ mit einer verzögerten Varianten des Signals $x$ korreliert, wie in Formel \ref{eq:XCorr} definiert. Der parameter $k$ wird als \emph{Lag} bezeichnet und gibt die Verzögerung an. 

\begin{equation}
\text{X-Corr}(x,y,k) = \sum_{n=-\infty}^{\infty} x[n-k] \cdot y[n]
\label{eq:XCorr}
\end{equation}

Im Prozess der so genannten \emph{Running Correlation} nutzt man die Cross-Correlation mit den Lags $k = 0 \cdots k_{max}$ zur Erstellung des \emph{Korrelationssignals} $r$, wie in Gleichung \ref{eq:runningCorrelation} definiert. Das Signal $r$ gibt Auskunft, zu welchen Verzögerungswerten $k$ die größten Ähnlichkeiten zwischen $x$ und $y$ gefunden wurden. 

\begin{equation}
r[k] = \text{X-Corr}(x,y,k) \quad, k = 0 , \ldots , k_{max} 
\label{eq:runningCorrelation}
\end{equation}

Abbildung \ref{img:slidingCorrelation} zeigt ein Beispiel für die Erzeugung von $r$ mit der Sliding Correlation. (A) zeigt das zu detektierende Signal $x$ und (B) das Signal $y$. (C) zeigt das Korrelationssignal $r$ mit den Lags $k = 1, \ldots ,1150$ \cite[S. 47 - 48]{dspMichigan}

\begin{figure}[h]
	\centering
	\includegraphics[width=0.5\textwidth]{bilder/slidingCorrelation.png}
	\caption{Beispiel einer Running Correlation}
	\label{img:slidingCorrelation}
\end{figure}

\section{Faltung}

Die \emph{Faltung} (engl. \emph{Convolution}) ist eine der Zentralen Operationen zwischen zwei Signalen, so wie die Addition oder die Mulitplikation. Sie wird mit dem Symbol $*$ notiert. Sie wird notiert mit $x* h = y$. 

Die Faltung basiert auf der \emph{Faltungs-Summe}, welche die Faltung zunächst Punktweise definiert. Die Gleichung wird in Formel \ref{eq:convolutionSum} abgebildet. In diesem Zusammenhang wird $x$ Eingangs- und $y$ als Ausgangs-Signal bezeichnet. Je nach Anwendungsfall bekommt $h$ den Namen \emph{Faltungs-Kernel}, \emph{Filter-Kernel} oder einfach \emph{Kernel}. \cite[S. 107-108]{dspGuide}

\begin{equation}
y[n] = x[n] * h[n] = \sum_{i=1}^{M} h[i] * x[n-i]
\label{eq:convolutionSum}
\end{equation}

Das Ergebnis einer Berechneten Faltungssumme nach \ref{eq:convolutionSum} ist ein einzelner Wert. Wird die Faltungs-Summe ähnlich der Cross-Correlation für $n = 1...N+M-1$ durchgeführt, ist das Ergebnis ein Signal. Die tatsächliche Faltung wird in Gleichung \ref{eq:convolution} definiert. $x$ ist ein Signal mit Support$(x) = [1,N]$ und Duration$(x) = N$, $h$ ist ein Signal mit Support$(x) = [1,M]$ und Duration$(h) = M$ und $y$ ist ein Signal mit Support$(y) = [1,N+M-1]$ und Duration$(y) = N+M-1$. Das heißt, dass das Eingangssignal um die Länge des Faltungskerns verlängert wird. Abbildung 	\ref{img:convolutionExample} zeigt ein Beispiel für die Faltung.\cite[S. 115-120]{dspGuide}

\begin{equation}
y = x * h = \big[ \: x[1] * h[1] , \ldots , x[N+M-1] * h[N+M-1] \: \big]
\label{eq:convolution}
\end{equation}

\begin{figure}[h]
	\centering
	\includegraphics[width=0.7\textwidth]{bilder/convolutionExample.png}
	\caption{Beispiel für die Faltung}
	\label{img:convolutionExample}
\end{figure}

Das neutrale Element der Faltung ist der \emph{Delta-Funktion}, definiert in Gleichung \ref{eq:delta} . Das heißt, dass $x * \delta = x$ . Die Faltung ist kommutativ, das heißt $ x * h = h * x = y$ . \cite[S. 107, 113 ]{dspGuide}

\begin{equation}
\delta[n] = 
\begin{cases}
1 \quad , n = 0\\
0 \quad ,  n \neq 0
\label{eq:delta}
\end{cases}
\end{equation}

Eines der wichtigsten Anwendungsgebiete der Faltung ist das Filtern. Weitere Erläuterungen werden in Kapitel \ref{sec:filter} gegeben.


\section{Diskrete Fourier-Transformation}

Die \emph{Fourier-Transformation} ist eine Familie von Transformationen, mit deren Hilfe Signale aus dem Zeit-Bereich in den Frequenz-Bereich transformiert werden. Das heißt, dass der unabhängige Parameter nach der Transformation nicht mehr die Zeit, sondern die Frequenz beschreibt. 

Die konkrete Berechnung der Transformation ist abhängig von den Eigenschaften des Signales. Die Variante, die die meiste Anwendung in der digitalen Signalverarbeitung findet, ist die \emph{Diskrete Fourier-Transformation} (kurz \textbf{DFT} ). Sie transformiert \emph{zeit-diskrete, periodische, unendliche} Signale (siehe Formel \ref{eq:time-disc-signal} und \ref{eq:periodicity}). Es exisitert sowohl eine reelle als auch eine complexe Variante der DFT. Die reelle Variante wird mit Hilfe reeller Zahlen, und die komplexe mit Hilfe komplexer Zahlen berechnet. An dieser Stelle werden beide Variante vorgestellt: Die komplexe, da der effizienteste Algorithmus zur Berechnung der DFT, die \emph{Fast-Fourier-Transformation} (\textbf{FFT}) auf ihr beruht, und die reelle, da sie das Verständnis der komplexen vereinfacht.\cite[S. 142 - 146]{dspGuide}

\subsection{reelle DFT}

Jedes zeitdiskretes, periodisches Signal kann erzeugt werden, indem eine endliche Anzahl von Cosinus- und Sinussignale geeigneter Frequenz und Amplitude aufaddiert werden. Der Umkehrschluss ist, dass sich jedes Signal in eine Menge von Sinus- und Cosinus-Signale zerlegen lässt, ohne das Information verloren geht. Diese Zerlegung wird als \emph{Dekomposition} bezeichnet. 

Die Sinus- und Cosinus-Wellen, die für die Dekomposition verwendt werden, werden als \emph{Basisfunktionen} bezeichnet. Die Basisfunktionen werden in Gleichung \ref{eq:baseFunctions} definiert. Abbildung \ref{img:fftExample01} visualisiert, wie ein Signal $x$ durch die Addition der vier Signale $\cos_1 + \cos_8 + \sin_4 + sin_15 = x$ zusammengesetzt (synthetisiert) wird.

\begin{equation}
\label{eq:baseFunctions}
\begin{split}
\cos_k[n] = \cos(2\pi k \frac{n}{N}) \\
\sin_k[n] = \sin(2\pi k \frac{n}{N})
\end{split}
\end{equation}

Für die Zerlegung eines $N$-Samples langen Signals $x$ werden höchstens $N+2$ Basisfunktionen benötigt, mit $\frac{N}{2}+1$ Cosinus- und  $\frac{N}{2}+1$ Sinus-Signale. Jede Basisfunktion hat die selbe Länge wie das zu synthetisierende Signal. Im Beispiel aus Abbildung \ref{img:fftExample01} haben die vier Basis-Funktionen als auch das Signal $x$ eine Länge von $N = 200$. Von den 202 benötigten Basisfunktionen werden in diesem Beispiel nur 4 benötigt, zwei Cosinus-Schwinungen mit $k = 1$ und $k = 8$ und zwei Sinus-Schwinungen mit $k = 2$ und $k=15$. Je höher der Parameter $k$, desto höher auch die Frequenz der Basisschwingung. 

\begin{figure}[h]
	\centering
	\includegraphics[width=1\textwidth]{bilder/fftExp01.png}
	\caption{Synthetisierung eines Signals $x$ aus vier Basis-Funktionen}
	\label{img:fftExample01}
\end{figure}

Die Addition aller Basis-Funktionen zur Komposition von $x$ wird auch als \emph{Synthese} oder \emph{inverse DFT} (\emph{iDFT}) bezeichnet. Sie wird in Gleichung  \ref{eq:inverseDFT} definiert. Das Signal $\ReX$ enthält die Amplituden für die $\cos$-Basisfunktionen und das Signal $\ImX$ die Amplituden für die $\sin$-Basisfunktionen. Dementsprechend haben die Längen $Duration(\ReX[x]) = Duration(\ImX[x]) = N+1$.

\begin{equation}
\begin{split}
x[n] = \sum_{k = 0}^{N/2}\ReX[k]\cos(2\pi k \frac{n}{N}) + \sum_{k = 0}^{N/2}\ImX[k]\sin(2\pi k \frac{n}{N}) \\
,n = 1 \ldots N = \text{Duration}(x)
\end{split}
\label{eq:inverseDFT}
\end{equation}

Abbildung \ref{img:reXimX} zeigt das $\ReX$ und das $\ImX$-Signal für das Beispiel aus Abbildung \ref{img:fftExample01}. Die Formel für die Synthese das Beispielsignal ist somit: $n = 1 \ldots 200: x[n] = 1 \cdot \cos(2\pi 1 \frac{n}{200}) + 2 \cdot \cos(2\pi 8 \frac{n}{200}) + 0.5 \cdot \sin(2\pi 2 \frac{n}{200}) + 1 \cdot \sin(2\pi 15 \frac{n}{200})  $

\begin{figure}[h]
	\centering
	\includegraphics[width=0.6\textwidth]{bilder/reXimX.png}
	\caption{ $\ReX$ und $\ImX$-Signal für das Beispiel aus Abbildung \ref{img:fftExample01}}
	\label{img:reXimX}
\end{figure}

Die Signale $\ReX$ und $\ImX$ bilden gemeinsam den sogenannten \emph{Frequenz-Bereich} und werden unter dem Variablennamen $X$ subsumiert. Wird $x$ als Eingangssignal gegeben, für das man den Frequenz-Bereich berechnen möchte, wird die \emph{forward DFT} nach Formel \ref{eq:forwardDFT} verwendet. Diese Operation wird auch kurz als \emph{DFT} bezeichnet. Wie sich erkennen lässt, handelt sich um eine Korrelation des Signales $x$ mit den einzelnen Basis-Funktionen. Je höher der Korrelations-wert der jeweiligen Basisfunktion, desto höher seine Amplitude und somit sein Beitrag zur späteren Synthese des ursprünglichen Signales. Abbildung \ref{img:timToFreq} fasst den Zusammenhang zwischen dem Zeit- und dem Frequenzbereich zusammen.

\begin{equation}
\begin{split}
\ReX'[k] = \sum_{n=0}^{N-1}x[n] \cos(2\pi k \frac{n}{N}) \\
\ImX'[k] = \sum_{n=0}^{N-1}x[n] \sin(2\pi k \frac{n}{N}) \\
k = 0, \ldots , N/2 \quad , N = \text{Duration}(x)
\end{split}
\label{eq:forwardDFT}
\end{equation}

\begin{figure}[h]
	\centering
	\includegraphics[width=0.7\textwidth]{bilder/timeToFrequency_self.png}
	\caption{Überblick über die DFT und die iDFT}
	\label{img:timToFreq}
\end{figure}

Es ist darauf zu achten, dass man durch die DFT die Signale $\ReX'$ und $\ImX'$ erhält, für die inverse DFT nach Formel \ref{eq:inverseDFT}  jedoch die Signale  $\ReX$ und $\ImX$ benötigt werden. Formel \ref{eq:reConversion} definiert die Umwandlung für das $\ReX$-Signal und Formel \ref{eq:imConversion} die Umwandlung des $\ImX$-Signal. Die Sonderfälle ds $\ReX$-Signals sind ein Nebenprodukt der reellen DFT, welche bei der komplexen DFT in Kapitel \ref{sec:comDFT} vereinfacht wird.

\begin{equation}
\ReX[k] = 
\begin{cases}
\frac{\ReX'[k]}{N} \; , \text{falls } k = 0 \\
\frac{\ReX'[k]}{N} \; , \text{falls } k = N/2 \\
\frac{\ReX'[k]}{N/2} \; ,\text{sonst} \\
\end{cases}
\label{eq:reConversion}
\end{equation}

\begin{equation}
\ImX[k] = \frac{\ImX'[k]}{N/2}
\label{eq:imConversion}
\end{equation}

Die Signale $ImX$ und $ReX$ werden zur Berechnung der DFT und iDFT verwendet, sind jedoch für den Menschen schwierig zu interpretieren. Sie stellen den Frequenz-Bereich in der sogenannten \emph{kartesischen Notation} (engl. \emph{rectagnular Notation}) dar. Um die Signale besser interpretieren zu können, werden sie in die \emph{polare Notation} transformiert. Dabei macht man sich zu Nutze, dass die Summe eines Sinus- und eines Cosinus-Signals der selben Frequenz ein Cosinus-Signal der selben Frequenz, jedoch mit veränderter Amplitude $M$ und Phase einer Phasenverschiebung $\phi$ erzeugt, wie aus Formel \ref{eq:sinToCosinus} hervorgeht.

\begin{equation}
\begin{split}
A \cos(x) + B \sin(x) = M \cos(x + \phi) \\
,M = \sqrt{A^2 + B^2} \;, \phi = arctan(B/A)
\end{split}
\label{eq:sinToCosinus}
\end{equation}


Aus dieser Umrechnungsvorschrift lässt sich die Transformation der kartesischen in die polare Notation ableiten. Sie wird in Formel \ref{eq:rectToPolar} definiert. Formel \ref{eq:polarToRect} definiert die Umkehroperation. Abbildung \ref{img:rectToPolar} visualisiert diese Konvertierung der Signale des Frequenz-Bereiches von kartesischer in polarer Notation anhand des Beispiels aus Abbildung \ref{img:reXimX}.

\begin{equation}
\begin{split}
\text{MagX}[k] = \sqrt{(\ReX'[k]^2 + \ImX'[k]^2 ) }\\
\text{PhaseX}[k] = \arctan{(\ImX'[k] / \ReX'[k]) }
\end{split}
\label{eq:rectToPolar}
\end{equation}

\begin{equation}
\begin{split}
\ReX'[k] = \text{MagX}[k] \cos(\text{PhaseX}[k])\\
\ImX'[k] = \text{MagX}[k] \sin(\text{PhaseX}[k])
\end{split}
\label{eq:polarToRect}
\end{equation}

\begin{figure}[h]
	\centering
	\includegraphics[width=1\textwidth]{bilder/rectToPolar.png}
	\caption{ Konvertierung des Frequenz-Bereiches aus Abbildung \ref{img:reXimX}  von kartesischer in polare Notation}
	\label{img:rectToPolar}
\end{figure}

Es gibt verschiedene Arten der Indexierung des \text{MagX}-Signals neben der einfachen Nummerierung $k = 0 \ldots N/2$

\subsection{Komplexe DFT}
\label{sec:comDFT}

\section{Filter}
\label{sec:filter}

\section{akustische Modellierung der menschlichen Stimme}
\section{Feststellung von Periodizität in Signalen}
\subsection{Zero-Crossing-Rate}
\subsection{Methoden des Frequenzbereiches}
\subsection{Autokorrelation}
\subsection{Cepstrum}