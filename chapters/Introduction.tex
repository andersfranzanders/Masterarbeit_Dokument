\chapter{Einleitung}

Die Schmerzbegutachtung bei Neugeborenen ist eine Aufgabe, die in vielen medizinischen Bereichen notwendig ist, angefangen bei der Überwachung von Routinemaßnahmen wie Impfungen bis hin zu Langzeitbeobachtungen über Wochen bis Monate zum Beispiel aufgrund schwerer Geburtsverläufe. Eine akkurate Schmerzbewertung ist dabei von hoher Wichtigkeit, sowohl für die Sicherung des Wohlbefindens des Babys, als auch für die Evaluation der Effektivität schmerzstillender Maßnahmen. Es gibt Hinweise darauf, dass insbesondere lang anhaltender Schmerz, welcher nicht korrekt erkannt und somit nicht behandelt wird, das weitere Schmerzempfinden des betroffenen Babys nachhaltig beeinflussen kann.\cite[S. 402 - 403]{PainAssessment03} \cite[S. 439]{PainAssessment01}

Babys äußern das Vorhandensein von Schmerz durch verschiedene Merkmale. Die für den Menschen am leichtesten wahrnehmbaren Kennzeichen sind beispielsweise ein verkrampfter Gesichtsausdruck oder Weinen. Auch physiologische Veränderungen, wie die Erhöhung der Herzfrequenz oder des Blutdruckes, lassen Rückschlüsse auf das Vorhandensein von Schmerz zu.\cite[S. 440]{PainAssessment01} Auf Basis dieser Indikatoren werden im klinischen Bereich zur Schmerzdiagnostik sogenannte \emph{Schmerz-Scales} eingesetzt. Eine Schmerz-Scale listet eine Reihe von Indikatoren, wie zum Beispiel den Blutdruck oder das Weinen, die im Zusammenhang mit Schmerz stehen. Für jedes Merkmal werden, je nach Grad der Ausprägung, Punkte vergeben, um schlussendlich den Schmerzgrad als \emph{Schmerz-Score} zu quantifizieren.\cite[S. 406]{PainAssessment03}

Insbesondere die Schmerzbewertung über längere Zeiträume stellt eine Herausforderung im klinischem Bereich dar. Die Arbeitsgruppe \glqq Laboratory for Biosignal Processing\grqq{} der Hochschule für Technik, Wirtschaft und Kultur Leipzig hat sich zum Ziel gesetzt, ein System zu entwickeln, welches die Schmerzbewertung bei Neugeborenen automatisiert und kontinuierlich vornimmt. Dazu sollen diverse Modalitäten, wie beispielsweise Video- und Audiodaten, ausgewertet werden. Das Ziel dieser Arbeit ist die Konzeption eines Teilaspektes dieses Systems, und zwar die automatisierte und kontinuierliche Schmerzdiagnostik anhand akustischer Signale. Der Schmerzgrad wird durch Schmerz-Scores bemessen. Bei einem Einsatz des geplanten System im klinischen Alltag würd eine größere Menge an Auswertungsergebnissen erzeugt werden, welche über den Schmerzgrad des überwachten Neugeborenen zu beliebigen Zeitpunkten Auskunft geben. Daher ist es wünschenswert, die Ergebnisse so zu visualisieren, dass eine Übersicht über den Schmerzverlauf leicht gewonnen werden kann. Ein weiteres Ziel dieser Arbeit ist deshalb die Erarbeitung eines Konzeptes zur Visualisierung der Schmerzbewertung in Bezug auf akustische Signale.

\section{Aufbau der Arbeit}

In \autoref{sec:foundations} wird zunächst eine Einführung in wichtige wissenschaftliche Grundlagen gegeben, die in dieser Arbeit Anwendung finden. Die Themenbereiche, welche besprochen werden, sind die Audiosignalverarbeitung mit dem Fokus auf die menschliche Stimme, Methoden der Schmerzdiagnostik bei Neugeborenen sowie Algorithmen des maschinellen Lernens. 

In \autoref{sec:concept} wird ein Überblick über das in dieser Arbeit entwickelte Konzept zur automatisierten Schmerzdiagnostik auf Basis akustischer Signale gegeben. Dazu werden zunächst die Anforderungen an das Konzept vorgestellt, ein Überblick über Veröffentlichungen mit ähnlichen Aufgabenstellungen gegeben und schlussendlich die einzelnen Verarbeitungsschritte umrissen. In den folgenden Kapitel werden diese Verarbeitungsschritte im Detail erläutert. 

%Nochmal lesen!
Der erste Schritt in der Verarbeitungskette ist die Detektion von Schreigeräuschen von Babys. Dazu werden in \autoref{sec:vad} Methoden zur Erkennung von Stimmaktivität vorgestellt und bezüglich ihrer Leistungsfähigkeit bei dem speziellen Anwendungsfall kindlicher Lautäußerungen evaluiert. Die Möglichkeiten zur automatisierten und kontinuierlichen Schmerzbewertung auf Basis der so erkannten Schreigeräusche werden in \autoref{sec:deduction} vorgestellt. Dazu werden Vorschriften zur Berechnung von Eigenschaften des Weines präsentiert, welche für die Schmerzdiagnostik genutzt werden können, sowie Möglichkeiten für den Entwurf von schmerzbezogenen Modellen diskutiert. In \autoref{sec:visualisation} wird abschließend das Visualisierungskonzept beschrieben.

\section{Fokus der Arbeit}

Der Fokus dieser Arbeit liegt auf der Konzipierung einer modular aufgebauten Verarbeitungskette, welche die kontinuierliche und automatisierte Schmerzbewertung sowie die Visualisierung dieser Bewertungsergebnisse ermöglicht. Der Verarbeitungsschritt, welcher am ausgiebigsten evaluiert wurde, ist die Erkennung der Stimmaktivität in \autoref{sec:vad}. Da alle darauf folgenden Verarbeitungsschritte umfassendere Datenerhebungen und Evaluationen in Zusammenarbeit mit medizinischen Fachkräfte erfordern, welche im Zeitrahmen dieser Arbeit nicht umsetzbar waren, werden hier begründete Vorschläge zur Aufgabenlösung gemacht, jedoch nicht evaluiert. Das Ziel dieser Arbeit ist insbesondere nicht die Erlangung neuer Erkenntnisse über biologische Prozesse in Zusammenhang mit der Schmerzwahrnehmung.