\chapter{Einleitung}

Die Schmerzdiagnostik Neugeborener ist eine Aufgabe, die in vielen medizinischen Kontexten notwendig ist: Angefangen von der Überwachung prozeduraler Eingriffe wie Impfungen, über die Thearipie von Entzugserscheinungen infolge des Drogenkonsums der Mutter bis hin zu Komplikationen bei der Geburt, welche die Beobachtung der Neugeborenen über mehrere Wochen bis Monate erfordert. Es gibt starke Hinweise darauf, dass insbesondere lang anhaltender Schmerz, welcher nicht korrekt erkannt und dessen Ursache somit nicht beseitigt wird, das Schmerzempfinden des Betroffenen bis in das Erwachsenenalter nachhaltig beeinflussen kann.\cite[S. 402]{PainAssessment03}

Babys äußern das Vorhandensein von Schmerz anhand einer Reihe von Merkmalen. Die für den Menschen am leichtesten sichtbaren Zeichen sind beispielsweise das Verkrampfen des Gesichtsausdruckes oder das Weinen. Aber auch physiologische Veränderungen wie der Erhöhung der Herzfrequenz oder des Blutdruckes lassen Rückschlüsse auf den Schmerzgrad zu.\cite[S. 440]{PainAssessment01} Auf Basis dieser Indikatoren werden im klinischen Alltag zur Schmerzdiagnostik sogenannte \emph{multimodale Pain Scales} eingesetzt. Eine Pain Scale listet eine Reihe von Merkmalen, wie zum Beispiel den Hautfarbe vor das Weinen, für die eine Korrelation mit Schmerz festgestellt wurde. Für jedes Merkmal werden, je nach Grad der Ausprägung, Punkte vergeben, um schlussendlich den Schmerzgrad als \emph{Pain Score} zu quantifizieren.\cite[S. 406]{PainAssessment03}

Forschungsbestrebungen der letzten Jahre haben sich zum Ziel gesetzt, die Schmerzbewertung zu automatisieren. Das Ziel dieser Arbeit ist die Entwicklung eines Konzeptes zur kontinuierlichen Schmerzbewertung bei Neugeborenen in einem multimodalen Verbund. Dabei soll der Schmerzgrad nicht nur abgeleitete, sondern auch visualisiert werden. Der Fokus liegt dabei auf der Analyse des Schmerzindikators \emph{Weinen}, das heisst der Analyse akustischer Signale. Ausgangspunkt aller Entwürfe und Überlegungen sind die über viele Jahre erprobten Pain Scales.

Die Lösung dieser Aufgabenstellung erfordert die Kombination verschiedener Wissenschaftsdisziplin, wie zum Beispiel der medizinischen Schreiforschung oder der akustischen Modellierung der menschlichen Stimme. Kapitel \ref{sec:foundations} bietet eine Einführung in die wichtigsten Grundlagen. In Kapitel \ref{sec:concept} wird ein Überblick über Veröffentlichungen mit ähnlichen Zielstellungen gegeben sowie dass in dieser Arbeit entwickelte Konzept grundlegend erläutert. Die folgenden Kapitel beleuchten die einzelnen Bausteine dieses Konzeptes genauer. In Kapitel \ref{sec:vad} werden Methoden zur Feststellung des Vorhandenseins von Weingeräuschen in einem akustischen Signal vorgestellt. Dazu werden klassische Methoden der Voice Activity Detection bezüglicher ihrer Leistungsfähigkeit im speziellen Anwendungsfall kindlicher Lautäußerungen eruiert und evaluiert. In Kapitel \ref{sec:acousticModel} werden Methoden entworfen, um aus den festgestellten Weingeräuschen den Schmerzgrad abzuleiten. Möglichkeiten der Visualisierung der Pain Score werden abschließend in Kapitel \ref{sec:visualisation} vorgestellt.

Diese Arbeit leistet vor allem den Entwurf eines modular aufgebauten und erweiterbaren Konzeptes, welches die kontinuierliche Feststellung und Visualisierung von Schmerz Scores ermöglicht. Das Modul, welches am ausgiebigsten evaluiert werden konnte, ist die Voice Activity Detection. Da alle darauf folgenden Module ausgiebigere Datenerhebungen in Zusammmenarbeit mit medizinischen Fachkräfte benötigen, wurden hier begründete Vorschläge zur Aufgabenbewältigung entworfen, welche jedoch nicht evaluiert werden konnten. Das Ziel dieser Arbeit war insbesondere nicht die Erlangung neuer Erkenntnisse über die tatsächliche Schmerzwahrnehmung Neugeborener.