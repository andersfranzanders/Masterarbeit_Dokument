\chapter{Einleitung}

Die Schmerzbegutachtung bei Neugeborenen ist eine Aufgabe, die in vielen medizinischen Bereichen notwendig ist, angefangen bei der Überwachung von Routinemaßnahmen wie Impfungen bis hin zu Langzeitbeobachtungen über Wochen oder Monate, zum Beispiel aufgrund schwerer Geburtsverläufe. Eine akkurate Schmerzbewertung sichert das Wohlbefinden des Babys und hilft bei der Evaluation der Effektivität schmerzstillender Maßnahmen. Es gibt Hinweise darauf, dass insbesondere lang anhaltender Schmerz, welcher nicht korrekt erkannt und somit nicht behandelt wird, das weitere Schmerzempfinden des Neugeborenen nachhaltig beeinflussen kann.\cite[S. 402 - 403]{PainAssessment03} \cite[S. 439]{PainAssessment01}

Babys äußern das Vorhandensein von Schmerz durch verschiedene Merkmale. Die für den Menschen am leichtesten wahrnehmbaren Kennzeichen sind beispielsweise ein verkrampfter Gesichtsausdruck oder Weinen. Auch physiologische Veränderungen, wie die Erhöhung der Herzfrequenz oder des Blutdruckes, lassen Rückschlüsse auf das Vorhandensein von Schmerz zu.\cite[S. 440]{PainAssessment01} Auf Basis dieser Indikatoren werden im klinischen Bereich sogenannte Schmerz-Scales eingesetzt. Diese ermöglichen die Schmerzbewertung, indem objektiv messbare Parameter mit subjektiven Eindrücken kombiniert werden, um die Schmerzintensität in Form von Schmerz-Scores zu quantifizieren.\cite[S. 100 - 101]{painInNeonates}

% Eine Schmerz-Scale listet eine Reihe von Indikatoren, wie zum Beispiel den Blutdruck oder das Weinen, die im Zusammenhang mit Schmerz stehen. Für jedes Merkmal werden, je nach Grad der Ausprägung, Punkte vergeben, um schlussendlich den Schmerzgrad als \emph{Schmerz-Score} zu quantifizieren.\cite[S. 406]{PainAssessment03}

Insbesondere die Schmerzbewertung über längere Zeiträume stellt eine Herausforderung im klinischen Bereich dar.\cite[S. 240]{painAssessmentStatus} Die Arbeitsgruppe \glqq Laboratory for Biosignal Processing\grqq{} der Hochschule für Technik, Wirtschaft und Kultur Leipzig hat sich zum Ziel gesetzt, ein System zu entwickeln, welches die Schmerzbewertung bei Neugeborenen automatisiert und kontinuierlich vornimmt. Dazu sollen diverse Modalitäten, wie beispielsweise Video- und Audiodaten, ausgewertet werden. Das Ziel dieser Arbeit ist die Konzeption eines Teilaspektes dieses Systems, welcher sich mit der automatisierten und kontinuierlichen Schmerzdiagnostik anhand akustischer Signale befasst. Die Bemessung des Schmerzgrades basiert auf den genannten Schmerz-Scores. Um die Auswertungsergebnisse der automatisierten Schmerzbewertung für das klinische Personal leicht überschaubar zu machen, ist ein weiteres Ziel die Erarbeitung eines Konzeptes zur Visualisierung der kontinuierlichen Schmerzdiagnose in Bezug auf akustische Signale.

%Bei einem Einsatz des geplanten System im klinischen Alltag würde eine größere Menge an Auswertungsergebnissen erzeugt werden, welche über den Schmerzgrad des überwachten Neugeborenen zu beliebigen Zeitpunkten Auskunft geben. Daher ist es wünschenswert, die Ergebnisse so zu visualisieren, dass eine Übersicht über den Schmerzverlauf leicht gewonnen werden kann. Ein weiteres Ziel dieser Arbeit ist deshalb die Erarbeitung eines Konzeptes zur Visualisierung der Schmerzbewertung in Bezug auf akustische Signale.

\section{Aufbau der Arbeit}

In \autoref{sec:foundations} wird eine Einführung in wichtige wissenschaftliche Grundlagen gegeben, die in dieser Arbeit Anwendung finden. Themenbereiche, welche besprochen werden, sind die Audiosignalverarbeitung mit Fokus auf der akustischen Modellierung der menschlichen Stimme, Methoden der Schmerzdiagnostik bei Neugeborenen sowie Algorithmen des maschinellen Lernens. 

In \autoref{sec:concept} wird ein Überblick über das in dieser Arbeit entwickelte Konzept zur automatisierten Schmerzbewertung auf Basis akustischer Signale gegeben. Dazu werden zunächst Veröffentlichungen mit ähnlichen Aufgabenstellungen vorgestellt. Anschließend werden die einzelnen Verarbeitungsschritte umrissen, welche in den folgenden Kapiteln detailliert erläutert werden. Weiterhin werden Möglichkeiten zur Eingliederung des Konzeptes in einen multimodalen Verbund beschrieben.

%Nochmal lesen!
Der erste Schritt in der Verarbeitungskette ist die Detektion von Schreigeräuschen von Babys. Hierzu werden in \autoref{sec:vad} Methoden zur Erkennung von Stimmaktivität in Audiosignalen vorgestellt und bezüglich ihrer Leistungsfähigkeit bei dem speziellen Anwendungsfall kindlicher Lautäußerungen evaluiert. Die Möglichkeiten zur automatisierten und kontinuierlichen Schmerzbewertung auf Basis der erkannten Schreigeräusche werden in \autoref{sec:deduction} vorgestellt. Dabei wird sowohl auf die Berechnung akustischer Merkmale des Weinens eingegangen, als auch Vorgehensweisen für den Entwurf schmerzbezogener Modelle diskutiert. Abschließend wird in \autoref{sec:visualisation} das Visualisierungskonzept vorgestellt.

\section{Fokus der Arbeit}

Der Fokus dieser Arbeit liegt auf der Konzipierung einer modular aufgebauten Verarbeitungskette, welche die kontinuierliche und automatisierte Schmerzbewertung sowie die Visualisierung der Bewertungsergebnisse ermöglicht. Das Ziel dieser Arbeit ist insbesondere nicht die Erlangung neuer Erkenntnisse über biologische Prozesse im Zusammenhang mit der Schmerzwahrnehmung. Der Verarbeitungsschritt, welcher am ausgiebigsten evaluiert wird, ist die Erkennung der Stimmaktivität in \autoref{sec:vad}, da sie die Grundlage für die folgenden Auswertungen darstellt. Da alle darauf folgenden Verarbeitungsschritte umfassendere Datenerhebungen und Evaluationen in Zusammenarbeit mit medizinischen Fachkräften erfordern, welche im Zeitrahmen dieser Arbeit nicht umsetzbar waren, werden hier begründete Vorschläge zur Aufgabenlösung präsentiert, jedoch nicht evaluiert.