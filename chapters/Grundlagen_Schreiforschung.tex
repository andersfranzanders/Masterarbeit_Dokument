\section{Physio-akustische Modellierung des Weinens}

\section{Schmer Scores}
Bei erwachsenen Menschen wird der Schmerzgrad typischerweise durch eine Selbsteinschätzung des Patienten unter der Leitung gezielter Fragen des Arztes vorgenommen. Bei Kindern unter 3 Jahren ist diese Selbsteinschätzung nicht möglich. Schmerz drückt sich in Veränderungen des psychologischen, körperlichen und biochemischen Verhaltens des Säuglings aus. Die für den Arzt am leichtesten feststellbaren Verhaltensänderungen sind von außen wahrnehmbaren Merkmale, wie zum Beispiel ein Verkrampfen des Gesichtsausdrucks, erhöhte Körperbewegungen oder lang anhaltendes Weinen. Um eine weitesgehend objektive Schmerzfeststellung zu ermöglichen, wurden sogenannte \emph{Pain-Scores} entwickelt, die durch ein Punktesystem den insgesamten Schmerzgrad des Babies quantifizieren.\cite{PainAssessment01} Es existieren \emph{eindimensionale} Pain-Scores, die den Schmerz nur Aufgrund der Beobachtung eines Merkmals beurteilen, so wie beispielsweise die reine Beurteilung des Gesichtsausdruckes. \emph{Mehrdimensionale} (auch \emph{ multimodale}) Pain-Scores beziehen mehrere Faktoren in das Scoring mit ein.\cite{PainAssessment02}. Tabelle \ref{tab:nips} zeigt das Scoring-System \glqq Neonatal Infant Pain Scale\grqq{}(NIPS) als Beispiel für eine multimodale Pain-Score. Der Säugling wird anhand der aufgeführten Kategorien bewertet und alle vergebenen Punkte aufsummiert. Ein insgesamter Wert von $>3$ zeigt Schmerz an, ein Wert von $>4$ großen Schmerz.\cite{nips}

\begin{table}[h]
	\footnotesize
	\centering
	\caption{NIPS-Scoring}
	\label{tab:nips}
	\begin{tabular}{@{}cccc@{}}
		\toprule
		\textbf{NIPS}     & \textbf{0 points} & \textbf{1 point}     & \textbf{2 points} \\ \midrule
		Facial Expr. & Relaxed           & Contracted           & -                 \\
		Cry               & Absent            & Mumbling             & Vigorous          \\
		Breathing         & Relaxed           & Different than basal & -                 \\
		Arms              & Relaxed           & flexed/stretched     & -                 \\
		Legs              & Relaxed           & flexed/stretched     & -                 \\
		Alertness         & Sleeping          & uncomfortable        & -                 \\ \bottomrule
	\end{tabular}
\end{table}


In den meisten mehrdimensionalen Scoring-Systeme werden die Schreigeräusche mit einbezogen. Tabelle \ref{tab:painscores} zeigt eine Übersicht über eine ausgewählte Menge an multimodalen Pain-Scores. Alle Pain-Scores sind für Kleinkinder bis 3 Jahren gedacht. In der Übersicht wird nicht wiedergegeben, welche weiteren Merkmale jeweils in das Scoring mit einbezogen werden, oder welche Insgesamtpunktzahlen auf welche Schmerzintensität hinweisen. Es soll an dieser Stelle nur verdeutlicht werden, welche unterschiedlichen Ansätze zur Bewertung des Schreiens aus medizinischer Sicht im Zusammenhang mit Pain-Scores existieren. Folgende Beobachtungen lassen sich aus der Übersicht ziehen:

1.) Die zu beobachtenden Eigenschaften des Weinens werden mit subjektiv behafteten Werten charakterisiert. Beispielsweise wird im N-PASS-System ist ein Schmerz-Schrei als \glqq High-pitched or silent-continuous crying\grqq{} beschrieben. Es wird nicht fest definiert, was als \glqq crying\grqq{} gilt oder welche Tonhöhe als \glqq high-pitched\grqq{} ist. Auch die Erstquellen geben keine festen Definitionen.

2.) Es gibt verschiedene Ansätze zur Bewertung des Weinens. Bei CRIE ist die Tonhöhe, bei BIIP die Länge und bei COMFORT die Art des Weinens entscheidend.

3.) Die Beschreibungen sind kurz und prägnant gehalten, der Arzt hat in keinem der Modelle auf mehr als drei Parameter des Schreiens zu achten. Die Begründung liegt darin, dass bei allen Modellen a.) das Schreien nur eines von mehreren Faktoren ist, und b.) Die Schmerzbestimmung in einem vorgegebenen Zeitrahmen durchführbare sein muss.

\begin{table}[h]
	\centering
	\caption{Übersicht über Pain-Scores}
	\label{tab:painscores}
	\begin{tabular}{@{}lll@{}}
		\toprule
		\textbf{System} & \textbf{P.} & \textbf{Description}                                                                                \\ \midrule
		FLACC\cite{flacc}           & 0           & No cry (awake or asleep)                                                                            \\
		& 1           & Moans or whimpers; occasional complaint                                                             \\
		& 2           & \begin{tabular}[c]{@{}l@{}}Crying steadily, screams or sobs, \\ frequent complaints\end{tabular}    \\\midrule
		N-PASS\cite{npass}          & -2          & No cry with painful stimul                                                                          \\
		& -1          & \begin{tabular}[c]{@{}l@{}}Moans or cries minimally \\ with painful stimuli\end{tabular}            \\
		& 0           & Appropiate Crying                                                                                   \\
		& 1           & \begin{tabular}[c]{@{}l@{}}Irritable or Crying at Intervals.\\ Consolable\end{tabular}                                                        \\
		& 2           & \begin{tabular}[c]{@{}l@{}}High-pitched or silent-continuous crying. \\ Not consolable\end{tabular} \\\midrule
		BIIP\cite{BIIP}            & 0           & No Crying                                                                                           \\
		& 1           & Crying \textless 2 minutes                                                                          \\
		& 2           & Crying \textgreater 2 minutes                                                                       \\
		& 3           & Shrill Crying \textgreater 2 minutes                                                                \\\midrule
		CRIES\cite{cries}            & 0           & If no cry or cry which is not high pitched                                                          \\
		& 1           & \begin{tabular}[c]{@{}l@{}}If cry high pitched but baby \\ is easily consoled\end{tabular}          \\
		& 2           & \begin{tabular}[c]{@{}l@{}}If cry is high pitched and baby \\ is inconsolable\end{tabular}          \\\midrule
		COVERS\cite{covers}          & 0           & No Cry                                                                                              \\
		& 1           & High-Pitched or visibly crying                                                                      \\
		& 2           & Inconsolable or difficult to soothe                                                                 \\\midrule
		PAT\cite{pat}             & 0           & No Cry                                                                                              \\
		& 1           & Cry                                                                                                 \\\midrule
		DAN\cite{dan}             & 0           & Moans Briefly                                                                                       \\
		& 1           & Intermittent Crying                                                                                 \\
		& 2           & Long-Lasting Crying, Continuous howl                                                                \\\midrule
		COMFORT\cite{comfort}         & 0           & No crying                                                                                           \\
		& 1           & Sobbing or gasping                                                                                  \\
		& 2           & Moaning                                                                                             \\
		& 3           & Crying                                                                                              \\
		& 4           & Screaming                                                                                           \\\midrule
		MBPS\cite{mbps}            & 0           & Laughing or giggling                                                                                \\
		& 1           & Not Crying                                                                                          \\
		& 2           & \begin{tabular}[c]{@{}l@{}}Moaning quiet vocalizing gentle or \\ whimpering cry\end{tabular}        \\
		& 3           & Full lunged cry or sobbing                                                                          \\
		& 4           & Full lunged cry more than baseline cry                                                              \\ \bottomrule
	\end{tabular}
\end{table}
