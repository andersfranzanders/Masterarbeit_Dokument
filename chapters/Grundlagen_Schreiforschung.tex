\chapter{Grundlagen der medizinischen Schrei-Forschung}

\section{Pain-Scales}
\label{sec:painScores}

Schmerz wird definiert als eine \glqq eine unangenehme wahrnehmbare und emotionale Erfahrung im Zusammenhang mit tatsächlicher oder potentiellen Gewebsschäden\grqq{}. Abseits von dieser theoretischen Definition hat der Mensch ein intuitives Verständnis für Schmerz, da jeder ihn bereits erfahren hat. In der ersten Hälfte des 20sten Jahrhunderts war die vorherschende Meinung, dass Neugeborene keinen Schmerz empfinden können. Beispielsweise wurde ihnen nach Operationen keine Schmerzmittel verabreicht oder in einigen Fällen noch nicht einmal betäubt während der Operationen. Die aktuell vorherrschende Meinung ist, dass Neugeborene genau wie erwachsene Menschen in der Lage, Schmerz zu empfinden. Die freien Nervenenden, die in der Lage sind, physisiche Schäden am Körper festzustellen und im Gehirn ein Gefühl von Schmerz auszulösen, sind bei Neugerorenen ebenso wie bei Erwachsenen über den Körper verteilt. Die hormonelle Reaktion ist ebenfalls vergleichbar. \cite[S. 402]{PainAssessment03} \cite[S. 438]{PainAssessment01}

Die Gründe für Schmerz bei Neugeborenen sind divers. Sie reichen über physische Schäden, aufgrund von komplikationen bei der Geburt oder Gewalteinwirkungen, über Erkankungen, wie Kopfschmerzen oder Infektionen, bis hin zu therapeutischen Prozeduren, wie Injektionen oder Desinfektionen von Wunden.  Das Vorhandensein von Schmerz ist anhand diverser physiologischen, biochemischen, verhaltensbezogenen und psyhologischen Veränderungen messbar.\cite[S. 441]{PainAssessment01}

Schlussendlich ist Schmerz jedoch immernoch ein subjektives Empfinden, weshalb der Grad des Schmerzes bei Erwachsenen typischerweise durch eine Selbsteinschätzung des Patienten unter der Leitung gezielter Fragen des Arztes vorgenommen wird. Bei Kindern unter 3 Jahren ist diese Selbsteinschätzung nicht möglich. Diee Einschätzung wird daher von anderen Personen vorgenommen. Im klinischen kontext sind dies medizinische Fachkräfte, wie Ärzte, Krankenpfleger oder Geburtshelfer. Die von außen am leichtesten feststellbaren Schmerzäußerungen sind die verhaltensbasierten Merkmale, wie zum Beispiel ein Verkrampfen des Gesichtsausdrucks, erhöhte Körperbewegungen oder lang anhaltendes Weinen.\cite[S. 438]{PainAssessment01} Die Schmerzdiagnostik durch die beobachtende Person ist etwas inherent subjektives und wird beeinflusst von Faktoren wie Alter, Geschlecht, kulturellen Hintergrund, persönlichen Erfahrungen mit Schmerz etc.\cite[S. 3]{overview} Um die Schmerzdiagnostik objektiver zu gestalten, wurden daher sogenannte \emph{Pain-Scales} entwickelt, die durch ein Punktesystem den Schmerzgrad des Babies quantifizieren.\cite[S. 438 - 439]{PainAssessment01} Es existieren \emph{monomodale} oder \emph{unidimensionale} Pain-Scales, die den Schmerz nur Aufgrund der Beobachtung eines Merkmals beurteilen, so wie beispielsweise die reine Beurteilung des Gesichtsausdruckes. Ein Merkmal wird in diesem Zusammenhang als \emph{Schmerz-Indikator} bezeichnet. \emph{ Multimodale}) oder auch \emph{Multidimensionale} Pain-Scales beziehen mehrere Schmerzindikatoren in das Scoring mit ein.\cite[S. 69 - 71]{PainAssessment02}. Tabelle \ref{tab:nips} zeigt das Scoring-System \glqq Neonatal Infant Pain Scale\grqq{}(NIPS) als Beispiel für eine multimodale Pain-Scale. Der Säugling wird anhand der aufgeführten Kategorien bewertet und alle vergebenen Punkte aufsummiert. Ein insgesamter Wert von $>3$ zeigt Schmerz an, ein Wert von $>4$ großen Schmerz.\cite{nips}

\begin{table}[h]
	\footnotesize
	\centering
	\caption{Neonatal Infant Pain Scale \cite{nips}}
	\label{tab:nips}
	\begin{tabular}{@{}cccc@{}}
		\toprule
		\textbf{NIPS}     & \textbf{0 points} & \textbf{1 point}     & \textbf{2 points} \\ \midrule
		Facial Expr. & Relaxed           & Contracted           & -                 \\
		Cry               & Absent            & Mumbling             & Vigorous          \\
		Breathing         & Relaxed           & Different than basal & -                 \\
		Arms              & Relaxed           & flexed/stretched     & -                 \\
		Legs              & Relaxed           & flexed/stretched     & -                 \\
		Alertness         & Sleeping          & uncomfortable        & -                 \\ \bottomrule
	\end{tabular}
\end{table}


Nach dem Muster der NIPS existieren viele weitere Pain-Scales. Sie unterscheiden sich hinsichtlich der Schmerz-Indikatoren, die betrachtet werden, dem Punktesystem oder den konkreten Einsatzzweck, wie zum Beispiel die Schmerzdiagnostik während oder nach der Schmerz-verursachenden Prozedur. Die meisten davon ziehen das Weinen oder Schreien der Kinder mit ein. In der englischen Fachliteratur ist von \glqq Cry\grqq{} die Rede.\cite[S. 97 - 98]{painInNeonates} In dieser Arbeit wird \glqq Cry\grqq{} mit \glqq Weinen\grqq{} oder mit dem neutraleren Begriff \glqq kindliche Lautäußerungen\grqq{} übersetzt. In den meisten multimodalen Pain-Scales werden die Lautäußerungen mit einbezogen. Tabelle \ref{tab:painscores} zeigt eine Übersicht über einige multimodalen Pain-Scales. In der Übersicht wird nur der Teil wiedergegeben, der sich auf die Lautäußerungen bezieht. Es wird nicht gezeigt, welche weiteren Merkmale jeweils in das Scoring mit einbezogen werden, für welchen Altersbereich die Scale gedacht ist oder welches Scoring auf welche Schmerzintensität hinweist. Es soll an dieser Stelle nur verdeutlicht werden, welche unterschiedlichen Ansätze zur Bewertung des Weinens aus medizinischer Sicht im Zusammenhang mit Pain-Scales existieren. 

%\begin{table}[H]
%	\centering
\begin{longtable}{@{}lll@{}}
	
	%	\begin{tabular}{@{}lll@{}}
	\toprule
	\textbf{System} & \textbf{P.} & \textbf{Description}                                                                                \\ \midrule
	FLACC***\cite{flacc}           & 0           & No cry (awake or asleep)                                                                            \\
	& 1           & Moans or whimpers; occasional complaint                                                             \\
	& 2           & \begin{tabular}[c]{@{}l@{}}Crying steadily, screams or sobs, \\ frequent complaints\end{tabular}    \\\midrule
	N-PASS***\cite{npass}          & -2          & No cry with painful stimul                                                                          \\
	& -1          & \begin{tabular}[c]{@{}l@{}}Moans or cries minimally \\ with painful stimuli\end{tabular}            \\
	& 0           & Appropiate Crying                                                                                   \\
	& 1           & \begin{tabular}[c]{@{}l@{}}Irritable or Crying at Intervals.\\ Consolable\end{tabular}                                                        \\
	& 2           & \begin{tabular}[c]{@{}l@{}}High-pitched or silent-continuous crying. \\ Not consolable\end{tabular} \\\midrule
	BIIP\cite{BIIP}            & 0           & No Crying                                                                                           \\
	& 1           & Crying \textless 2 minutes                                                                          \\
	& 2           & Crying \textgreater 2 minutes                                                                       \\
	& 3           & Shrill Crying \textgreater 2 minutes                                                                \\\midrule
	CRIES*\cite{cries}            & 0           & If no cry or cry which is not high pitched                                                          \\
	& 1           & \begin{tabular}[c]{@{}l@{}}If cry high pitched but baby \\ is easily consoled\end{tabular}          \\
	& 2           & \begin{tabular}[c]{@{}l@{}}If cry is high pitched and baby \\ is inconsolable\end{tabular}          \\\midrule
	COVERS**\cite{covers}          & 0           & No Cry                                                                                              \\
	& 1           & High-Pitched or visibly crying                                                                      \\
	& 2           & Inconsolable or difficult to soothe                                                                 \\\midrule
	PAT*\cite{pat}             & 0           & No Cry                                                                                              \\
	& 1           & Cry                                                                                                 \\\midrule
	DAN**\cite{dan}             & 0           & Moans Briefly                                                                                       \\
	& 1           & Intermittent Crying                                                                                 \\
	& 2           & Long-Lasting Crying, Continuous howl                                                                \\\midrule
	COMFORT*\cite{comfort}         & 0           & No crying                                                                                           \\
	& 1           & Sobbing or gasping                                                                                  \\
	& 2           & Moaning                                                                                             \\
	& 3           & Crying                                                                                              \\
	& 4           & Screaming                                                                                           \\\midrule
	MBPS\cite{mbps}            & 0           & Laughing or giggling                                                                                \\
	& 1           & Not Crying                                                                                          \\
	& 2           & \begin{tabular}[c]{@{}l@{}}Moaning quiet vocalizing gentle or \\ whimpering cry\end{tabular}        \\
	& 3           & Full lunged cry or sobbing                                                                          \\
	& 4           & Full lunged cry more than baseline cry                                                              \\ \bottomrule
	%\end{tabular}
	\caption{Übersicht über Pain-Scales. Legende zu den Einsatzbereichen: *** Anhaltender/chronischer Schmerz, ** Prozeduraler Schmerz, *Post-Operativer Schmerz\cite[S. 98 ]{painInNeonates} }
	\label{tab:painscores}
\end{longtable}
%\end{table}

Da die Begriffe \emph{Pain-Scale} und \emph{Pain-Score} in einigen Veröffentlichungen inkonsistent verwendet werden, wird in dieser Arbeit die Konvention getroffen, dass mit \emph{Pain-Scale} das System zur Schmerzdiangostik gemeint ist, und mit \emph{Pain-Score} die auf Basis der Pain-Scale vergebene Punktzahl.

Aus der Übersicht in Tabelle \ref{tab:painscores} lassen sich die folgenden Beobachtungen schließen:

\begin{enumerate}

\item Die Eigenschaften der Lautäußerungen werden zum größten Teil mit \emph{subjektiv behafteten Begriffen} beschrieben. Beispielsweise wird im N-PASS-System ist ein Schmerz-Schrei als \glqq High-pitched or silent-continuous crying\grqq{} beschrieben. Dabei werden die Begriffe \glqq High-pitched\grqq{} und \glqq silent-continuous\grqq{} nicht näher definiert.  Auch die Anleitungen der entsprechenden Pain-Scales werden keine festen Definitionen gegeben. Die BIIP nutzt als einzige Scale objektiv messbare Eigenschaften. Dies erleichtert den praktischen Einsatz der Pain-Scales, führt jedoch zu einem Interpretationsspielraum und somit zu einem von der diagnostizierenden Person abhängigen Scoring.

\item Verschiedene Scales basieren die ableitung des Schmerzgrades auf verschiedenen Kriterien. Bei CRIE ist die Tonhöhe, bei BIIP die Länge und bei COMFORT die Art des Weinens entscheidend.

\item Die Beschreibungen sind kurz und prägnant gehalten, die diagnostizierende Person hat in keinem der Modelle auf mehr als drei Eigenschaften des Schreiens zu achten.
\end{enumerate}


\section{Schmerz-Schrei aus medizinischer Sicht}

An dieser Stelle stellt sich der Leser eventuell die Frage, woher die unterschiedlichen Bewertungskriterien in den verschiedenen Schmerz-Scales stammen. Gibt es eine Pain-Scale, die \glqq mehr recht hat\grqq  als andere? Dafür sind zuerst zwei grundlegendere Fragen zu beantworten:

\begin{enumerate}
	 \item Ist es möglich, aus den akustischen Eigenschaften den motivierenden Grund für die Lautäußerung abzuleiten?  Klingt ein Hunger-Weinen anders als ein Schmerz-Weinen?
	 \item Ist es möglich, anhand der akustischen Eigenschaften den Schweregrad dieses motivierenden Grundes abzuleiten?
\end{enumerate}

Die Annahme, dass es möglich ist, aus dem Schreien den Grund abzuleiten, wird als \glqq Cry-Types Hypothesis\grqq{} bezeichnet. Die berühmtesten Befürworter dieser Hypothese ist eine skandinavische Forschungsgruppe, auch bezeichnet als \glqq Scandinavian Cry-Group\grqq , die diese Idee in dem Buch \glqq Infant Crying: Theoretical and Research Perspectives\grqq \cite{crygroup} publizierte machte. Die Annahme ist, dass die verschiedenen Ursachen \emph{Hunger, Freude, Schmerz, Geburt und Anderes} klare Unterschiede hinsichtlich ihrer akustischen Merkmale aufweisen, welche am Spectogramm ablesbar seien. Wenige einige Jahre Später zeigten Müller et al \cite{cryisnoise}, dass bei leichter Veränderung der Bedingungen der Experimente die Unterscheidung nicht mehr möglicht ist. Die Gegenhypothese ist, dass Weinen \glqq nichts als undifferenziertes Rauschen\grqq{} sei. 50 Jahre später liegt kein anerkannter Beweis für die eine oder andere Hypothese vor. Es gibt lediglich starke Hinweise dafür, dass die Plötzlichkeit des Eintretens des Grundes sich in den akutischen Eigenschaften bemerkbar macht. Ein plötzliches Ereignis, wie ein Nadelstich oder ein lautes Geräuch, führen auch zu einem plötzlich beginnenden Schreien. Ein langsam einretendes Ereignis, wie ein langsam zunehmender chmerz oder lHunger führen auch zu einem langsam eintretenden Weinen. Da nach Kenntniss des Autors bis heute keine wissenschaftlich belastbarer Beweis vorgelegt wurde, wird empfohlen, den Grund aus dem Kontext abzuleiten.\cite[S. 9 - 13, 17 - 19]{signal}

Die Zweite Frage nach der Ableitung der Stärke des Unwohlseins aus den akustischen Eigenschaften des Geschreis wird in der Fachliteratur unter dem Begriff \emph{Cry as a graded Signal} subsumiert. Je \glqq stärker\grqq{} das Weinen, desto höher das Unwohlsein (\emph{Level of Distress (LoD)}) des Säuglings. Tatsächlich bemessen wird dabei der von dem Beobachter vermutete Grad des Unwohlsein des Babies, und nicht der tatsächliche Grad, da dieser ohne die Möglichkeit der direkten Befragung des Kindes nie mit absoluter Sicherheit bestimmt werden kann. Ein hohes Level of Distress hat vor allem eine schnelle Reaktion der Aufsichtspersonen zur Beruhigung des Babies zur Folge, womit dem Geschrei eine Art Alarm-Funktion zukommt. Es gibt starke Hinweise darauf, dass das Level of Distress anhand objektiv messbarer Eigenschaften des Audiosignals bestimmt werden kann. So herrscht beispielsweise weitesgehend Einigung darüber, dass ein \glqq lang\grqq{} anhaltendesWein auf einen hohen Level of Distress hinweist. Insofern aus dem Kontext des Schreiens Schmerz als wahrscheinlichste Ursache eingegrenzt werden kann, kann aus einem hohen Level of Distress ein hoher Schmerz abgeleitet werden. \cite[S. 13 - 17]{signal} \cite{lod} Es herrscht wiederum keine Einigung darüber, welche akustischen Eigenschaften im Detail ein hohes Level of Distress anzeigen. Carlo V Bellieni et al \cite{dan} haben festgestellt, dass bei sehr hohem Schmerz in Bezug auf die DAN-Scala (siehe Tabelle \ref{tab:painscores}) die Tonhöhe steigt. Qiaobing Xie et al \cite{lod} haben festgestellt, dass häufiges und \glqq verzerrtes\grqq{} Schreien (ohne feststellbares Grundfrequenz, da der Ton stimmlos erzeugt wird)  auf einen hohen Level of Distress hinweist.

\section{Klassische Schreiforschung}

Das Wissenschaftsgebiet, welches sich aus medizinischer Sicht mit der Analyse und Interpretation von Lautäußerungen auseinandersetzt, wird als \grqq Schrei-Forschung\glqq{} bezeichnet. Das bis heute wohl prominenteste Schreiforschungs-Team ist die vergangenen Kapitel erwähnte \glqq Scandinavian Cry-Group\grqq \cite{crygroup}, welche seit den 60er Jahren die Laute von Babies systematisch erforschten. Das hauptsächliche Werkzeug zur Analyse der Lautäußerungen war das Spektogramm, vorgestellt in Kapitel \ref{sec:theVoice}. Das Spektogramm wurde damals durch analoge Technologien hergestellt, wobei das Spektogramm buchstäblich auf ein Stück Papier gebrannt wurde.  Das Ziel war es, Muster  in diesen Spektogrammen zu erkennen, die abnormale von normalen Weinen unterscheiden, um beispielsweise Krankheiten erkennen zu können. \cite[S. 142]{signal} Teil der Scandinavian Cry-Group waren H Golub und M Corwin, die in der Veröffentlichung \glqq A Physioacoustic Model of the Infant Cry \grqq{} \cite{cryModel} ein Vokabular zur Beschreibung typischer, im Spektogramm erkennbarer Muster kindlicher Lautäußerungen festgelegt haben. Da das Vokabular bis heute Einsatz findet, werden wichtige Teilbereiche an dieser Stelle vorgestellt. Außerdem werden Begriffe eingeführt, die von Zeskind et al in \glqq Rythmic organization of the Sound of Infant Cry \grqq{} veröffentlicht wurden.\cite{rythmic}

\subsection{Phyisio-Akustische Modellierung des Weinens}
\label{sec:acousticModel}

Das Weinen von Babies lässt sich im allgemeinen als das \glqq rythmische Wiederholen eines beim ausatmen erzeugen Geräusches, einer kurzen Pause, einem Einatmungs-Geräusch, einer zweiten Pause, und dem erneuten Beginnen des Ausatmungs-Geräusches.\grqq beschreiben. \cite{wolff}.

Folgende grundlegenden Begriffe werden definiert. Sie werden in Abbildung \ref{img:cryVocabulary} veranschaulicht.

\begin{itemize}
	 \item \textbf{Expiration:} Der Klang, der bei einem einzelnen, ununterbrochenem Ausatmen mit Aktivierung der Stimmbänder durch das Baby erzeugt wird. \cite{rythmic}. Der von Golub et al \cite[S. 61]{cryModel} verwendete Begriff \textbf{Cry-Unit} wird in dieser Arbeit synonym verwendet. Umgangssprachlich ist handelt es sich um einen einzelnen, ununterbrochenen \emph{Schrei}.
	\item \textbf{Inspiration:} Der Klang, der beim Einatmen durch das Baby erzeugt wird.
	\item  \textbf{Burst:} Die Einheit von einer Expiration und der darauf folgenden Inspiration. Das heisst, dass die zeitliche Dauer eines Bursts sowohl das Expiration-Geräusch, das Inspiration-Geräusch als auch die beiden Pausen zwischen diesen Geräuschen umfasst. Praktisch ergibt sich das Problem, dass vor allem bei stärkerem Hintergrundrauschen die Inspiration-Geräusche häufig weder hörbar noch auf dem Spektrogramm erkennbar sind. Daher wird die Zeitdauer eines Bursts oder Cry-Unit vom Beginn einer Expiration bis zum Beginn der darauf folgenden Expiration definiert und somit allein von den Expirations auf die Bursts geschlossen. Implizit wird somit eine Inspiration zwischen zwei Expirations angenommen.
	\item  \textbf{Cry:} Die insgesamte klangliche Antwort zu einem spezifischen Stimulus. Eine Gruppe mehrerer Cry-Units.\cite[S. 61]{cryModel} In dieser Arbeit wird ein \emph{Cry} auch als \textbf{Cry-Segment} bezeichnet, um Verwechslungen zu vermeiden.
\end{itemize}

\begin{figure}
	\centering
	\includegraphics[width=0.7\textwidth]{bilder/cryVoc02.png}
	\caption{Veranschaulichung des Grundvokabulars}
	\label{img:cryVocabulary}
\end{figure}

Cry-Units werden von H Golub und M Corwin in eine der drei folgenen Kategorien eingeordnet, bezeichnet als \emph{Cry-Types}: \cite[S. 61 - 62]{cryModel}

\begin{itemize}
	 \item \textbf{Phonation} beschreibt eine Cry-Unit mit einer \glqq vollen Vibration der Stimmbänder\grqq{} und einer Grundfrequenz zwischen 250 und \SI{700}{\hertz}. Entspricht umgangssprachlich einem Weinen mit einem \glqq klaren, hörbaren Ton\grqq{}.
	 \textbf{Hyper-Phonation} beschreibt eine Cry-Unit mit einer \glqq falsetto-artigem Vibration der Stimmbänder\grqq{} mit einer Grundfrequenz zwischen 1000 und \SI{2000}{\hertz}. Entspricht umgangssprachlich einem Weinen mit einem \glqq sehr hohen, aber klar hörbaren Ton\grqq{}.
	 \textbf{Dysphonation:} beschreibt eine Cry-Unit ohne klar feststellbare Tonhöhe, produziert durch Turbulenzen an den Stimmbändern. Entspricht umgangsprachlichen dem \glqq Brüllen oder Krächzen\grqq{}.
\end{itemize}

Die folgenden weiteren Eigenschaften werden für einzelne Cry-Units extrahiert. Die hier gezeigte Liste ist eine Kombination von Features, die in verschiedenen Veröffentlichungen eingeführt wurden.

\begin{itemize}
	\item \textbf{Duration:} Die zeitliche Dauer der Cry-Unit.
	\item \textbf{Duration of Inspiration: }Die zeitliche Dauer der Pause bis zur nächsten Cry-Unit.
	\item \textbf{Grundfrequenz:} der Cry-Unit. Für eine Cry-Unit kann die durchschnittliche, die höchste, niedrigste, und Varianz der Grundfrequenz berechnet werden.
	\item \textbf{Frequenz der Formanten:} einer Cry-Unit. Wie bei der Grundfrequenz kann der Durchschnitt, das Maximum, Minimum etc. für eine Cry-Unit berechnet werden.
	\item \textbf{Ratio2: } Verhältnis zwischen den Energien der Frequenzen unterhalb von \SI{2000}{\hertz} und oberhalb von \SI{2000}{\hertz}
	\item \textbf{Cry-Mode Changes:} Häufigkeit des Wechsels des Cry-Modes innerhalb einer Cry-Unit.
	\item \textbf{Amplitude:} Die Lautstärke der Cry-Unit, gemessen in Dezibel. \cite[S. 85]{parentalPerception} \cite[S. 156]{threeCryTypes}
\end{itemize}

Golub et al haben weiterhin eine Reihe von Features betrachtet, die das zeitliche Verhalten der Grundfrequenz und der harmonischen Obertöne innheralb einer Cry-Unit beschreiben. \cite[S. 73]{cryModel}

\begin{itemize}
\item \textbf{Pitch of Shift:} Grundfrequenz nach einem schnellen Anstieg zu Beginn der Cry-Unit
\item \textbf{Glide:} Kurzes, starkes ansteigen der Grundfrequenz
\item  \textbf{Glottal Roll:} Dysphonation, die häufig am Ende einer Cry-Unit nach einem Abfall der Grundfrequenz beobachtet wird.
\item  \textbf{Vibrato:} Mehr als vier starke Schwankungen der Grundfrequenz innerhalb einer Cry-Unit.
\item  \textbf{Melody-Type:} einer Cry-Unit. Meist: fallend, steigend/fallend, steigend, fallend/steigend, flach. 
\item  \textbf{Continuity:} Verhältnis zwischen stimmhaften und nicht-stimmhaften Bereichen der Cry-Unit
\item  \textbf{Double Harmonic Break:} Das Aufkommen einer zweiten Serie von harmonischen Obertönen zwischen den eigentlichen harmonischen Obertönen der Cry-Unit.
\item  \textbf{Biphonation:} Das Aufkommen einer zweiten Grundfrequenz eigener harmonischen Obertönen zusätzlich zu der eigentlichen Grundfrequenz.
\item  \textbf{Noise Concentration:} Starke Energiespitzen zwischen 2000 und \SI{2300}{\hertz}
\item  \textbf{Furcation:} Plötzliches Aufteilen der Grundfrequenz und harmonsichen Obertöne in mehrere, schwächeren.
\end{itemize}

Abbildung \ref{img:cryMelodies} visualsiert diese Grundfrequenz bezogenen Features in einem schematisch dargstellten Spektogramm.

\begin{figure}
	\centering
	\includegraphics[width=0.7\textwidth]{bilder/melodyTypes.png}
	\caption{(1) Pitch of Shift (2) Maximale Grundfrequenz (3) Minimum der Grundfrequenz (4) Biphonation (5) Double Harmonic Break (6) Vibrato (7) Glide (8) Furcation \cite[S. 142]{signal}}
	\label{img:cryMelodies}
\end{figure}

Die folgende Features in Bezug auf das gesamte Cry-Segment, oder zumindest einer Reihe mehrerer Cry-Units berechnet:

\begin{itemize}
	\item \textbf{Cry Latenca: } Zeit zwischen Stimulus, wie zum Beispiel einem Nadelstich, und erster Cry-Unit
	\item \textbf{Utterances: } Anzahl der Cry-Units im Segment
	\item \textbf{Short Utterances: } Anzahl stimmloser Cry-Units im Segment
	\item .... und statistische Auswertungen bezüglich aller oben genannten Cry-Unit bezogenen Features, wie beispielsweise der Durchschnitt aller durchschnittlichen Tonhöhen, Anzahl des Vorkommens bestimmter Melodiekonturen, Varianz der Länge von Cry-Units etc.\cite[S. 85]{parentalPerception}
\end{itemize}

Verschiedene Krankheitsbilder wurden in Zusammenhang mit dem vorkommen Cry-Segment bezogener Features gebracht. So wurde eine Korrelation zwischen dem Anstieg der durchschnittlichen Grundfrequenz, häufiger Biphonation und geringer Duration in Zusammenhang mit Gehirnschäden gebracht. Tendenziell niedrige Grundfrequenzen korrellieren Trisomy 13, 18 und 21\cite[S. 85]{parentalPerception}

\subsection{Diskussion}
\label{sec:cryDiscussion}

Bis heute bleibt die Analyse von kindlichen Lautäußerungen weitesgehend unstandartisiert \cite[S. 142]{signal}:
\begin{itemize}
\item Es gibt keine Einigung darüber, welche der zahlreichen vorgestellten Eigenschaften die wichtigsten sind. Beispielsweise konzentrierten sich Golub et al \cite{cryModel} vermehrt auf die Erkennung von Mustern im Melodieverlauf, Zeskind et al auf zeitliche Eigenschaften. \cite{rythmic}. Die Eigenschaft, die am häufigste mit Schmerz, Krankheitsbildern und sonstigen Abnormalitäten in Verbindung gebracht wird, ist eine untypisch hohe oder niedrige Tonhöhe. Bei einigen Features, die vorallem von Golub et al verwendet wurden \cite{cryModel}, ist nicht einmal gesichert, ob es sich nicht doch um technische Artefakte der damals verwendeten Analogtechnik handelt \cite[S. 84 - 85]{parentalPerception}
\item Zusammenhänge, die zwischen bestimmten Eigenschaften der kindlichen Lautäußerung und Krankheitsbildern festgestellt wurden, haben häufig eine gute Specificity, aber schlechte Sensitivity. So wurde zum Beispiel festgestellt, dass Kinder, die an plötzlichen Kindstot verstarben, fast immer eine Erhöhung der Frequenz des ersten Formanten in Verbindung mit häufigen Cry-Mode-Changes zeigten. Es zeigen jedoch ebenfalls viele Babys die gleichen Charakteristiken in Bezug auf ihre Weingeräusche, ohne zu sterben. Die größere Herausforderung scheint somit zu sein, Features zu finden, die nicht die Specificity, sondern die Senstivity erhöhen.\cite[S. 85]{parentalPerception}
\item Selbst, wenn in verschiedenen Studien das selbe Feature verwendet wird, wie zum Beispiel die durschchnittliche Tonhöhe, ist nicht standartisiert, wie diese exakt zu berechnen ist. Mit \glqq durchschnittliche Tonhöhe des Segmentes\grqq{} kann gemeint sein: (1) die Durchschnittliche Tonhöhe, errechnet aus den durchschnittlichen Tonhöhen der der Cry-Units (2) Die durchschnittlcihe Tonhöhe aller festgestellten Tonhöhen (3) die durchschnittliche Tonhöhe nur von Ausatmungslauten etc.
\item Golub et al behaupten, bereits in den achziger Jahren ein System zur computergestützten und voll automatisierten Analyse von Cry-Segmenten implementiert zu haben. Das System nimmt (1.) eine Audioaufnahme, gespeichert auf einer Kasette an, (2.) berechnet Formanten, Grundfrequenz und Amplitude gegen die Zeit, (3.) sampled die Grundfrequenz-Kontur (4.) berechnet insgesamt 88. akkumulierte Features für das gesamte Segment und (5.) zieht Schlussfolgerungen aus den 88 Features, wie zum Beispiel das Vorhandensein einer bestimmten Krankheit.\cite[S. 75 - 76]{cryModel} Abseits der kurzen Erwähnung der Existenz dieser \grqq Mutter aller automatisierten Analysesysteme für das Weinen von Babys\grqq{} konnte der Autor dieser Arbeit keine Implementierungsdetails oder sonstige genaueren Ausführungen finden, welche für diese Arbeit von höchstem Interesse gewesen wären.
\end{itemize}


