\chapter{Visualisierung}
\label{sec:visualisation}

Zur Erläuterung der Visualisierung werden die in Tabelle \ref{tab:fictional_painscales_viz} aufgeführten fiktiven Pain Scales definiert. 

\begin{table}[h]
\footnotesize
\centering
\caption{Fiktive Pain Scales zur Erläuterung der Visualisierung}
\label{tab:fictional_painscales_viz}
\begin{tabular}{lllll}
\hline
Score     & \multicolumn{2}{c}{\glqq Length-Scale\grqq}                    & \multicolumn{2}{c}{\glqq Max-Scale\grqq}                                           \\ 
         & Subjektive Krit. & Objektive Krit.                  & Subjektive Krit.   & Objektive Krit.                                    \\ \hline
0 Punkte & kein Weinen      & Außerhalb Segment                & kein Weinen        &                                                    \\
1 Punkt  & kurzes Weinen    & S-Length$(cs) \leq \SI{1}{\minute}$ & seichtes Weinen    & $\SI{0}{\second} < min_{cu}(cs) \leq \SI{1}{\second}$ \\
2 Punkte & langes Weinen    & S-Length$(cs) > \SI{1}{\minute}$ & stärkeres Weinen   & $\SI{1}{\second} < min_{cu}(cs) \leq \SI{2}{\second}$ \\
3 Punkte & -                & -                                & energisches Weinen & $min_{cu}(cs) > \SI{2}{\second}$                   \\ \hline
\end{tabular}
\end{table}

Gleichung 


