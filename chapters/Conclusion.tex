\chapter{Fazit und Dikussion}

Das Ziel dieser Arbeit war die Erarbeitung eines Konzeptes zur automatisierten Schmerzbegutachtung am Beispiel von Audioaufnahmen Neugeborener sowie die Entwicklung eines entsprechenden Visualisierungskonzeptes. Zur Erfüllung dieser Aufgabe wurde eine modular aufgebaute Verarbeitungskette entworfen, welche als Input ein Audiosignal entgegen nimmt, den Schmerzverlauf ermittelt und schlussendlich visualisiert. Für alle Verarbeitungsschungschritte wurden verschiedene Lösungsmöglichkeiten präsentiert.

Der erste dieser Verarbeitungsschritte war die Detektion von Schreigeräuschen in einem Audiosignal. Hierzu wurden verschiedene Methoden zur automatisierten Erkennung von Stimmaktivität bezüglich ihrer Performance in Anwendung auf Babystimmen evaluiert. Die besten Ergebnisse erzielte eine auf dem Cepstrum basierende Methode, welche eine Genauigkeit von 88,98\% für einen Testdatensatz mit einem Signal-Rausch-Abstand von \SI{3}{\decibel} erreichte. Daraufhin wurde beschrieben, wie die erkannten Schreigeräusche genutzt werden können, um eine kontinuierliche Schmerzbegutachtung durchzuführen. Grundlage dafür ist die Unterteilung des Audiosignals in Zeitbereiche, die für die Schmerzbewertung herangezogen werden. Hierfür wurde ein Algorithmus zur Gruppierung von Schreigeräuschen vorgestellt, welcher nahe beieinander liegende Schreigeräusche zu Segmenten zusammenfasst. Für eine automatisierte Schmerzbewertung ist es weiterhin notwendig, verschiedene akustische Merkmale des Weinens aus dem Signal zu extrahieren. Dazu wurde eine Reihe an Berechnungsvorschriften präsentiert, welche sich vor allem auf den Zeitbereich beziehen. Weiterhin wurde ein auf ordinaler Regression basierendes Vorgehen entworfen, um die im klinischen Alltag eingesetzten Schmerz-Scales, welche vor Allem bei der Bewertung des Weinens auf subjektiven Beschreibungen beruhen, zu operationalisieren und so das Weinen automatisiert auf Schmerz-Scores abzubilden. Schlussendlich wurde ein Visualisierungskonzept vorgestellt, welches den zeitlichen Verlauf der Schmerz-Scores mit Hilfe einer intuitiven Farbsemantik darstellt. Das Visualisierungskonzept lässt sich ebenfalls für die Eingliederung in einen multimodalen Verbund verwenden.

Jeder Schritt der Verarbeitungskette ist isoliert betrachtet komplex genug, um als Ausgangspunkt zukünftiger Forschungen zu dienen. Beispielsweise bezieht das vorgestellte Verfahren zur Erkennung der Stimmaktivität Informationen des zeitlichen Verlaufes der Stimme nur im geringen Maße mit ein. Zur weiteren Erhöhung der Erkennungsgenauigkeit können Klassifizierungsalgorithmen erprobt werden, die auf zeitlich veränderliche Signale spezialisiert sind, wie zum Beispiel \emph{Hidden Markov Models} oder \emph{Zeitverzögerte Neuronale Netze}. Auch bereits etablierte Algorithmen zur Erkennung von Stimmaktivität, die in Bereichen wie dem Mobilfunk Anwendung finden, können für die Detektion von Babystimmen evaluiert werden. Davon abgesehen kann es im klinischen Alltag notwendig sein, neben dem Hintergrundrauschen weitere Störgeräusche, wie beispielsweise Stimmen von Erwachsenen, robust Babystimmen zu trennen.

Alle darauf folgenden Verarbeitungsschritte bedürfen weitreichenderen Evaluationen in Zusammenarbeit mit medizinischen Fachkräften. So ist das vorgestellte Verfahren zur Gruppierung von Schreigeräuschen ist nur eines von mehreren, denkbaren Vorgehen, um Zeitbereiche innerhalb eines längeren Audiosignals zu finden, auf die bei der Schmerzbewertung Bezug genommen wird. Der präsentierte Algorithmus müsste gegen Audiosignale getestet werden, die von Langezeitüberwachungen Neugeborener stammen. Medizinische Fachkräfte müssten daraufhin evaluieren, ob ihnen die von dem Algorithmus festgelegten Gruppierungen von Schreigeräuschen sinnvoll erscheinen. Je nach Ergebnis dieser Evaluation müssen eventuell alternative Algorithmen zum Finden schmerzbezogener Zeitbereiche entwickelt werden.

Das Vorgehen zur Operationalisierung der Schmerz-Scales konnte im zeitlichen Rahmen dieser Arbeit nur konzipiert, aber nicht durchgeführt werden. Eine tatsächliche Umsetzung in Zusammenarbeit mit medizinischen Fachkräften wird sowohl zeigen, wie praktikable das geplante Vorgehen ist, als auch inwiefern sich die beschriebenen akustischen Eigenschaften eignen, um Schmerz-Scores automatisiert zu prognostizieren. Die vorgestellte Menge an Eigenschaften muss voraussichtlich um Attribute des Frequenzbereiches und des Melodieverlaufes ergänzt werden. Es ist ebenfalls möglich, dass Eigenschaften eines weitaus größeren Zeitbereiches akkumuliert werden müssen, um  Schmerz akkurat bewerten zu können.

Das Visualisierungskonzept konzentriert sich auf die Darstellung des zeitlichen Verlaufes der Schmerz-Scores. Da die Visualisierung vor allem als Hilfsmittel zur Veranschaulichung der Auswertungsergebnisse gedacht ist, müsste in Zusammenarbeit mit medizinischen Fachkräften evaluiert werden, inwiefern sie adäquat für den klinischen Einsatz ist. So ist beispielsweise denkbar, dass anstatt der Darstellung des zeitlichen Verlaufes die Visualisierung akkumulierter Auswertungsergebnisse gewünscht ist.

Das weitreichendere Ziel ist, das vorgestellte Konzept dahingehend weiterzuentwickeln, dass es in einem System implementiert werden kann und robust genug für den praktischen Einsatz im klinischen Alltag wird. In Kombination mit weiteren Modulen zur Auswertung zusätzlicher Modalitäten, wie beispielsweise Videodaten, kann so ein System zur kontinuierlichen, multimodalen Schmerzbegutachtung geschaffen werden. Ein solches System wird einen umfassenden Beitrag zur Verbesserung der Betreuungssituation Neugeborener leisten, indem Schmerzepisoden auch bei Abwesenheit des medizinischen Personals sicher festgestellt und protokolliert werden können. Insofern steht zu hoffen, dass dieses Forschungsthema weiter verfolgt wird und schlussendlich in den klinischen Alltag integriert werden kann.

%Das Bauchgefühl des behandelnden Artzes wird eventuell eine wichtige Variable bleiben, die nicht berechnet werden kann.


%Ziel dieser Arbeit war die Erarbeitung eines Konzeptes zur Ableitung von Schmerz Scores aus akustischen Signalen sowie die Visualisierung der Schmerz Scores zum kontinuierlichen Monitoring von Neugeborenen. Dazu wurde in Kapitel \ref{sec:concept} zunächst eine Übersicht über die Verarbeitungspipeline gegeben. In diesem Kapitel werden die Ergebnisse der einzelnen Verarbeitungsschritte, sowie mögliche Erweiterungen und Optimierungen vorgestellt, an denen zukünftige Forschungen ansetzen können.

%Der erste Schritt bei der Analyse der Audiosignale ist die Detektion der Stimme von Babys. In Kapitel \ref{sec:vad} wurden verschiedene Methoden der Voice Activity Detection evaluiert. Die besten Ergebnisse wurden mit Hilfe von Eigenschaften des \glqq Cepstrum\grqq -Bereiches erzielt. Für einen Testdatensatz mit einem Signal/Rausch-Abstand von \SI{3}{\decibel} wurde so eine Klassifizierungsgenauigkeit von 88.98\% erreicht. Zukünftige Forschungen können an dieser Stelle ansetzen und die Klassifizierungsgenauigkeit weiter erhöhen, indem Informationen über den zeitlichen Verlauf stärker zur Klassifizierung genutzt werden. Die Entscheidung über das Vorhandensein von Stimme wird in der vorgestellten Methode für jedes \SI{25}{\milli\second} lange Signalfenster isoliert getroffen. Ob jedoch eine Reihe als stimmhaft erkannte Signalfenster ingesamt ein Geräusch ergibt, bei welchem es sich tatsächlich um den Wein-Laut eines Babys handelt, wird nicht mit in Betracht gezogen. So ist es momentan noch nicht möglich, eine Aufnahme eines sprechenden Erwachsenen mit hoher Stimmlage von der Aufnahme eines Babys zu unterscheiden. Es wird empfohlen, zeitbezogene Klassifizierungsalgorithmen, wie zum Beispiel \emph{Hidden Markov Models} oder \emph{Zeitverzögerte Neuronale Netze} zu erproben, um die Klassifizierungsgenaugikeit weiter zu steigern.

%In Kapitel \ref{sec:deduction} wurden Konzepte vorgestellt, um den Schmerzgrad aus Audiosignalen abzuleiten. Es wurde ein datengetriebener Ansatz entworfen, der die Zusammenarbeit mit medizinischen Fachkräften erfordert. Es müsste zunächst eine Datenbank mit Audioaufnahmen von Babys erstellt werden. Daraufhin würde man, mit der Unterstützung medizinischer Fachkräfte, den Schmerzgrad in Form von Pain Scores für die Audioaufnahmen feststellen. Das Ergebnis wäre ein gelabelter Trainingsdatensatz, welcher daraufhin genutzt werden kann, um mit Hilfe geeigneter Klassifizierungs- oder Regressionsalgorithmen Prädiktoren zu entwerfen, welche die Schmerzdiagnostik anhand objektiv messbarer Signaleigenschaften ermöglichen. Da es im Zeitrahmen dieser Arbeit jedoch nicht möglich gewesen ist, einen solchen Trainingsdatensatz in Zusammenarbeit mit medizinischen Fachkräften zu erstellen, können zukünftige Forschungen den hier vorgezeichneten Plan umsetzen. Darüber hinaus eignet sich dieses Vorgehen nicht nur, um Prädiktoren auf Basis von Pain Scales zu erzeugen. Es könnten ebenso Prädiktoren für das intuitive Schmerzempfinden von Geburtshelfern oder Müttern entworfen werden.

%Abschließend wurde in Kapitel \ref{sec:visualisation} ein Konzept zur Visualisierung des Schmerz Scores vorgestellt. Es wurde vorgeschlagen, den zeitlichen Verlauf des Schmerz Scores durch eine farbliche Codierung nach einem Ampelschema zu visualisieren. Es wurde vorgestellt, wie dieses Ampelschema für Pain Scales mit beliebig abgestuften Scorings angepasst wird. Es ist im Zeitrahmen dieser Arbeit jedoch nicht möglich gewesen, das Visualisierungskonzept in Zusammenarbeit mit medizinischen Fachkräften bezüglich seiner Praxistauglichkeit zu evaluieren. Es wird empfohlen, in weiteren Forschungsbestrebungen in Zusammenarbeit mit medizinischen Fachkräften weitere Visualisierungskonzepte zu eruierten und umzusetzen.

%Wie zu sehen ist, bietet allein die Umsetzung der Schmerzbewertung auf Basis akustischer Signale viel Raum für zukünftige Forschungen. Das größere Ziel bleibt die multimodale Schmerzbewertung. Dazu müssen Methoden zur kontinuierlichen Schmerzdiagnostik weiterer Schmerzdindikatoren, wie zum Beispiel dem Gesichtsausdruck, erforscht und schlussendlich zu einem multimodalen System kombiniert werden. [knackiges Abschlusstatement]