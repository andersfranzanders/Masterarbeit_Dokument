\chapter{Fazit und Diskussion}

Das Ziel dieser Arbeit war der Entwurf eines Konzeptes zur automatisierten Schmerzbegutachtung Neugeborener am Beispiel von Audioaufnahmen sowie die Entwicklung eines entsprechenden Visualisierungskonzeptes. Zur Erfüllung dieser Aufgabe wurde eine modular aufgebaute Verarbeitungskette ausgearbeitet, welche als Input ein Audiosignal entgegennimmt, den Schmerzverlauf ermittelt und schlussendlich visualisiert. Für alle Verarbeitungsschungschritte wurden verschiedene Lösungsmöglichkeiten präsentiert und gegeneinander abgewogen.

Der erste dieser Verarbeitungsschritte war die Detektion von Schreigeräuschen in einem Audiosignal. Hierzu wurden verschiedene Methoden zur automatisierten Erkennung von Stimmaktivität bezüglich ihrer Performance in der Anwendung auf Babystimmen evaluiert. Die besten Ergebnisse erzielte ein auf dem Cepstrum basierendes Verfahren, welches eine Genauigkeit von 88,98\% für einen Testdatensatz mit einem Signal-Rausch-Abstand von \SI{3}{\decibel} erreichte. Im Folgenden wurde beschrieben, wie die erkannten Schreigeräusche genutzt werden können, um eine kontinuierliche Schmerzbegutachtung durchzuführen. Grundlage dafür war die automatisierte Unterteilung des Audiosignals in Zeitbereiche, die für die Schmerzbewertung herangezogen werden. Hierfür wurde ein Algorithmus zur Gruppierung von Schreigeräuschen vorgestellt, welcher nahe beieinander liegende Schreigeräusche zu Segmenten zusammenfasst. Für eine automatisierte Schmerzbewertung ist es weiterhin notwendig, verschiedene akustische Merkmale des Weinens aus dem Signal zu extrahieren. Dazu wurde eine Reihe an Berechnungsvorschriften präsentiert, welche sich vor allem auf den Zeitbereich beziehen. Weiterhin wurde ein auf ordinaler Regression basierendes Vorgehen entworfen, um die Beurteilungskriterien der im klinischen Alltag eingesetzten Schmerz-Scales, welche vor allem bei der Bewertung des Weinens auf subjektiven Beschreibungen beruhen, zu operationalisieren und so die Signalbereiche automatisiert auf Schmerz-Scores abzubilden. Schlussendlich wurde ein Visualisierungskonzept vorgestellt, welches den zeitlichen Verlauf der Schmerz-Scores mit Hilfe einer intuitiven Farbsemantik darstellt. Das Visualisierungskonzept lässt sich ebenfalls in einen multimodalen Verbund eingliedern.

Jeder Schritt der Verarbeitungskette kann als Ausgangspunkt zukünftiger Forschungen dienen. Beispielsweise bezieht das vorgestellte Verfahren zur Erkennung der Stimmaktivität Informationen des zeitlichen Verlaufes der Stimme nur im geringen Maße mit ein. Zur weiteren Erhöhung der Erkennungsgenauigkeit können Klassifizierungsalgorithmen erprobt werden, die auf zeitlich veränderliche Signale spezialisiert sind, wie zum Beispiel Hidden Markov Models oder Zeitverzögerte Neuronale Netze. Auch bereits etablierte Algorithmen zur Erkennung von Stimmaktivität, die in Bereichen wie dem Mobilfunk Anwendung finden, können für die Detektion von Babystimmen evaluiert werden. Davon abgesehen kann es im klinischen Einsatz notwendig sein, neben dem Hintergrundrauschen weitere Störgeräusche, wie beispielsweise Stimmen von Erwachsenen, robust von den Babystimmen zu trennen.

Alle darauf aufbauenden Verarbeitungsschritte bedürfen weitreichenderen Evaluationen in Zusammenarbeit mit medizinischen Fachkräften. So ist das vorgestellte Verfahren zur Gruppierung von Schreigeräuschen nur eines von mehreren denkbaren, um Zeitbereiche innerhalb eines längeren Audiosignals zu finden, auf welche bei der Schmerzbewertung Bezug genommen wird. Zur Evaluation muss der präsentierte Algorithmus mit Audiosignalen getestet werden, die bei Langzeitüberwachungen Neugeborener aufgenommen wurden. Medizinische Fachkräfte müssen daraufhin bewerten, ob ihnen die von dem Algorithmus festgelegten Gruppierungen von Schreigeräuschen sinnvoll erscheinen. Je nach Ergebnis dieser Evaluation ist es eventuell notwendig, alternative Algorithmen zum Finden schmerzbezogener Zeitbereiche zu entwickeln.

Das Vorgehen zur Operationalisierung der Bewertungskriterien von Schmerz-Scales konnte im zeitlichen Rahmen dieser Arbeit nur konzipiert, aber nicht durchgeführt werden. Eine tatsächliche Umsetzung in Zusammenarbeit mit medizinischen Fachkräften wird zeigen, inwiefern sich die beschriebenen akustischen Eigenschaften zur automatisierten Berechnung von Schmerz-Scores eignen. Die vorgestellte Menge an Eigenschaften muss voraussichtlich um Attribute des Frequenzbereiches und des Melodieverlaufes ergänzt werden. Es ist ebenfalls möglich, dass Eigenschaften eines weitaus größeren Zeitbereiches berücksichtigt werden müssen, um den Schmerzgrad akkurat bewerten zu können.

Das Visualisierungskonzept konzentriert sich auf die Darstellung des zeitlichen Verlaufes der Schmerz-Scores. Da die Visualisierung vor allem der Veranschaulichung der Auswertungsergebnisse dient, muss in Zusammenarbeit mit medizinischen Fachkräften evaluiert werden, inwiefern sie adäquat für den klinischen Einsatz ist. So ist beispielsweise denkbar, dass anstatt der Darstellung des zeitlichen Verlaufes die Visualisierung akkumulierter Auswertungsergebnisse eine höhere Priorität hat.

Das weitreichendere Ziel ist, das vorgestellte Konzept so zu vervollständigen, dass es robust genug für den praktischen Einsatz im klinischen Alltag ist und als Prototyp implementiert werden kann. In Kombination mit weiteren Komponenten zur Auswertung zusätzlicher Modalitäten, wie beispielsweise Videodaten, kann so ein System zur kontinuierlichen, multimodalen Schmerzbegutachtung geschaffen werden. Ein solches System wird einen umfassenden Beitrag zur Verbesserung der Betreuungssituation Neugeborener leisten, indem Schmerzepisoden auch bei Abwesenheit des medizinischen Personals festgestellt und protokolliert werden. Insofern steht zu hoffen, dass die Ergebnisse dieser Arbeit in den klinischen Alltag integriert werden können und Anstoß für weitere Forschung geben.