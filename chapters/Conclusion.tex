\chapter{Zusammenfassung}

Ziel dieser Arbeit war die Erarbeitung eines Konzeptes zur Ableitung von Schmerz Scores aus akustischen Signalen sowie die Visualisierung der Schmerz Scores zum kontinuierlichen Monitoring von Neugeborenen. Dazu wurde in Kapitel \ref{sec:concept} zunächst eine Übersicht über die Verarbeitungspipeline gegeben. In diesem Kapitel werden die Ergebnisse der einzelnen Verarbeitungsschritte, sowie mögliche Erweiterungen und Optimierungen vorgestellt, an denen zukünftige Forschungen ansetzen können.

Der erste Schritt bei der Analyse der Audiosignale ist die Detektion der Stimme von Babys. In Kapitel \ref{sec:vad} wurden verschiedene Methoden der Voice Activity Detection evaluiert. Die besten Ergebnisse wurden mit Hilfe von Eigenschaften des \glqq Cepstrum\grqq -Bereiches erzielt. Für einen Testdatensatz mit einem Signal/Rausch-Abstand von \SI{3}{\decibel} wurde so eine Klassifizierungsgenauigkeit von 88.98\% erreicht. Zukünftige Forschungen können an dieser Stelle ansetzen und die Klassifizierungsgenauigkeit weiter erhöhen, indem Informationen über den zeitlichen Verlauf stärker zur Klassifizierung genutzt werden. Die Entscheidung über das Vorhandensein von Stimme wird in der vorgestellten Methode für jedes \SI{25}{\milli\second} lange Signalfenster isoliert getroffen. Ob jedoch eine Reihe als stimmhaft erkannte Signalfenster ingesamt ein Geräusch ergibt, bei welchem es sich tatsächlich um den Wein-Laut eines Babys handelt, wird nicht mit in Betracht gezogen. So ist es momentan noch nicht möglich, eine Aufnahme eines sprechenden Erwachsenen mit hoher Stimmlage von der Aufnahme eines Babys zu unterscheiden. Es wird empfohlen, zeitbezogene Klassifizierungsalgorithmen, wie zum Beispiel \emph{Hidden Markov Models} oder \emph{Zeitverzögerte Neuronale Netze} zu erproben, um die Klassifizierungsgenaugikeit weiter zu steigern.

In Kapitel \ref{sec:deduction} wurden Konzepte vorgestellt, um den Schmerzgrad aus Audiosignalen abzuleiten. Es wurde ein datengetriebener Ansatz entworfen, der die Zusammenarbeit mit medizinischen Fachkräften erfordert. Es müsste zunächst eine Datenbank mit Audioaufnahmen von Babys erstellt werden. Daraufhin würde man, mit der Unterstützung medizinischer Fachkräfte, den Schmerzgrad in Form von Pain Scores für die Audioaufnahmen feststellen. Das Ergebnis wäre ein gelabelter Trainingsdatensatz, welcher daraufhin genutzt werden kann, um mit Hilfe geeigneter Klassifizierungs- oder Regressionsalgorithmen Prädiktoren zu entwerfen, welche die Schmerzdiagnostik anhand objektiv messbarer Signaleigenschaften ermöglichen. Da es im Zeitrahmen dieser Arbeit jedoch nicht möglich gewesen ist, einen solchen Trainingsdatensatz in Zusammenarbeit mit medizinischen Fachkräften zu erstellen, können zukünftige Forschungen den hier vorgezeichneten Plan umsetzen. Darüber hinaus eignet sich dieses Vorgehen nicht nur, um Prädiktoren auf Basis von Pain Scales zu erzeugen. Es könnten ebenso Prädiktoren für das intuitive Schmerzempfinden von Geburtshelfern oder Müttern entworfen werden.

Abschließend wurde in Kapitel \ref{sec:visualisation} ein Konzept zur Visualisierung des Schmerz Scores vorgestellt. Es wurde vorgeschlagen, den zeitlichen Verlauf des Schmerz Scores durch eine farbliche Codierung nach einem Ampelschema zu visualisieren. Es wurde vorgestellt, wie dieses Ampelschema für Pain Scales mit beliebig abgestuften Scorings angepasst wird. Es ist im Zeitrahmen dieser Arbeit jedoch nicht möglich gewesen, das Visualisierungskonzept in Zusammenarbeit mit medizinischen Fachkräften bezüglich seiner Praxistauglichkeit zu evaluieren. Es wird empfohlen, in weiteren Forschungsbestrebungen in Zusammenarbeit mit medizinischen Fachkräften weitere Visualisierungskonzepte zu eruierten und umzusetzen.

Wie zu sehen ist, bietet allein die Umsetzung der Schmerzbewertung auf Basis akustischer Signale viel Raum für zukünftige Forschungen. Das größere Ziel bleibt die multimodale Schmerzbewertung. Dazu müssen Methoden zur kontinuierlichen Schmerzdiagnostik weiterer Schmerzdindikatoren, wie zum Beispiel dem Gesichtsausdruck, erforscht und schlussendlich zu einem multimodalen System kombiniert werden. [knackiges Abschlusstatement]